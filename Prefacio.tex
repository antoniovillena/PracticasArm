%%%%%%%%%%%%%%%%%%%%%%%%%%%%%%%%%%%%%%%%%%%%%%%%%%%%%%%%%%%%%%%%%%%
%%% Documento LaTeX                                             %%%
%%%%%%%%%%%%%%%%%%%%%%%%%%%%%%%%%%%%%%%%%%%%%%%%%%%%%%%%%%%%%%%%%%%
% Título:   Introducción
% Autor:    Ignacio Moreno Doblas
% Fecha:    2014-02-01
% Versión:  0.5.0
%%%%%%%%%%%%%%%%%%%%%%%%%%%%%%%%%%%%%%%%%%%%%%%%%%%%%%%%%%%%%%%%%%%

\chapterbeginx{Prefacio}
\minitoc

Este manual recoge el conjunto de prácticas que se desarrollan en la
primera parte de la asignatura  ``Estructura de Computadores'',
asignatura que se imparte en el segundo curso de la ETSI Informática
de la Universidad de Málaga, en las titulaciones de Ingeniería en Informática,
Ingeniería Técnica en Informática de Sistemas, e Ingeniería Técnica en
Informática de Gestión. El objetivo de estas prácticas, y por lo tanto
el de este manual, es por un lado cubrir el nivel intermedio entre el
 lenguaje de alto nivel y el lenguaje máquina, y por otro lado,
utilizando como herramienta el lenguaje ensamblador, presentar los
mecanismos básicos que utiliza el subsistema de entrada salida de un
computador para la sincronización y transferencia de la
información. Como prototipo de lenguaje ensamblador hemos elegido la
arquitectura del i8086 puesto que de 'facto' se ha convertido en el
estándar en el  mercado de computadores personales.


El seguimiento de los contenidos del manual requiere algunos
conocimientos previos de la arquitectura de un computador básico y 
experiencia en programación con lenguajes de alto nivel, conceptos que
el alumno conoce pues ha cursado las asignaturas de  ``Tecnología de
Computadores'' y ``Elementos de Programación'' en el primer curso. A su vez,
las prácticas que se proponen en el manual permiten que el alumno
consiga un manejo suficiente de un computador a bajo nivel, lo que
proporciona un conjunto de
conocimientos claves para abordar con éxito las asignaturas de
``Sistemas Operativos I y II'' o ``Equipos Periféricos''.

 
El manual se ha dividido en 6 capítulos, cada uno de los cuales
corresponde a una práctica. Las prácticas no son independientes,
puesto que se han organizado de manera secuencial: la realización de
una práctica aportará los conocimientos necesarios para poder abordar
satisfactoriamente las siguientes. A continuación, revisaremos los
objetivos específicos de cada práctica.

Así, en la práctica 1, se describen brevemente las características
generales del i8086. Se repasan los conceptos de registros, {\it flags} y la
representación de números naturales y enteros. También se explica
brevemente el funcionamiento de algunas instrucciones básicas del
i8086 y sobre todo, se presenta  el entorno de trabajo: el
ensamblador, el enlazador y el depurador. 


En la práctica 2 nos centramos en explicar cómo se representan los
datos en la memoria del computador, así como la traducción a lenguaje
ensamblador de sentencias y estructuras de control de alto nivel. Por
ejemplo, veremos cómo se expresa en ensamblador cada uno de los tipos básicos
de datos de alto nivel que el alumno ya conoce (incluyendo vectores y
matrices).
                            
A continuación, en la práctica 3, describimos cómo se gestionan a bajo
nivel las subrutinas (procedimientos y funciones de alto
nivel). Veremos cómo se implementan las diferentes fases de gestión de
una subrutina y programaremos subrutinas en ensamblador a partir de
una traducción sistemática del código en alto nivel.

Con la práctica 4 comenzamos a manejar la E/S del computador.
Empezamos presentando los mecanismos básicos que utiliza un computador
tipo PC para comunicarse con los periféricos: puertos de E/S, concepto
de controlador, concepto de interrupción y sistema de interrupciones
en el i8086. A continuación, utilizando como ejemplo un sencillo dispositivo
conectado al puerto paralelo,
explicaremos cómo se programan los distintos parámetros del entorno de
las interrupciones (controlador de interrupciones, los puertos de E/S
del controlador del puerto paralelo, instalación de un vector de
interrupción y de una nueva rutina de tratamiento de interrupción)
para atender cualquier interrupción hardware que se produzca desde el
puerto paralelo.

 
En la práctica 5 se estudia el método de E/S controlada por programa.
Nos centraremos en el funcionamiento de tres periféricos: la pantalla,
el teclado y el ratón. También se presentarán ejemplos del uso de algunas
rutinas del sistema (llamadas generalmente interrupciones software) que
nos permitirán controlar cómodamente estos dispositivos.
 
En la práctica 6, estudiaremos los aspectos relacionados con la
temporización a bajo nivel en un computador tipo PC. El control de la
medida de tiempo es fundamental en aquellas aplicaciones que se basen
en la atención de eventos periódicos (reloj del sistema,
interrupciones periódicas del sistema operativo, generación de
sonidos, ...).

En este manual las prácticas están organizadas en dos grandes
apartados: un apartado de {\it Lectura previa} y un apartado de {\it
  Enunciados de la práctica}. El apartado de {\it Lectura previa} recoge
la información básica que el alumno necesita y que se recomienda a
éste que lea antes de ir al laboratorio. El apartado de {\it
  Enunciados de la práctica} incluye el conjunto de ejercicios y
prácticas que el alumno desarrollará en el laboratorio.

\newpage
Queremos comentar que aquellos  programas que el alumno puede necesitar durante la
realización de las prácticas, se proporcionan en el laboratorio, así
como en la página:

\vspace{2mm} 
\noindent{\bf \footnotesize
http://www.ac.uma.es/educacion/cursos/informatica/EstComp/}
\vspace{1mm}                                                    

En esa dirección se encuentra además más información relacionada con
el resto de los contenidos de la asignatura de ``Estructura de
Computadores''.

Para finalizar esta presentación queremos expresar nuestro
agradecimiento a los profesores del departamento de Arquitectura de
Computadores de la Universidad Politécnica de Cataluña (UPC), que
amablemente nos cedieron parte del material que utilizan en su
docencia de la asignatura de ``Estructura de Computadores I'',
impartida en la titulación de Ingeniería Informática de la UPC, y que
nos sirvió para organizar algunas de las prácticas que presentamos
aquí. Así mismo, queremos agradecer al resto de los profesores del
departamento de Arquitectura de Computadores de la Universidad de
Málaga su inestimable ayuda por sus comentarios y sugerencias en la
elaboración inicial de las prácticas, especialmente a Rafael Asenjo
Plaza, Francisco Corbera Peña, José Mª Linares González, Manuel
Ujaldón Martínez y Emilio Zapata. Nuestro más sincero agradecimiento a
todos ellos.
 
\vspace{0.5cm}
 
\hfill{Málaga, Marzo de 2002}   

\vspace{0.8cm}

\hfill{\large \bf Los Autores}
\chapterend
