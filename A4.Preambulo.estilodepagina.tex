%%%%%%%%%%%%%%%%%%%%%%%%%%%%%%%%%%%%%%%%%%%%%%%%%%%%%%%%%%%%%%%%%%%
% Tabla de materias:
%--------------------%
% 1 Márgenes de página
%%%%%%%%%%%%%%%%%%%%%%%
% Para conocer los parámetros de diseño de páginas, tales como
%  los márgenes izquierdo, derecho, anchura de página, etc.
%  véase el archivo ``Page layout.png'' que acompaña esta plantilla.
% Así se conocerá mejor cómo adaptar el documento según los 
%  requisitos del usuario.

% 1 Márgenes de página
%-------------------------------%
% Parámetros de estilo de página.
% DIN A4: 29.7 cm x 21 cm
%   área neta: 3 cm + 3 cm + 15 cm.
%
% Definición de márgenes de página
%  even para páginas pares
%  odd  para páginas impares
\newlength{\realoddsidemargin}    % \oddsidemargin menos 1 in.
\newlength{\realevensidemargin}   % \evensidemargin menos 1 in.
\newlength{\realtopmargin}        % \topmargin menos 1 in.
%
% Asignación de márgenes de página
% ASIGNESE en caso de querer cambiarlo
\setlength{\realtopmargin}{2cm}     % REAL top margin.
\setlength{\realoddsidemargin}{3cm}   % REAL oddside margin.
\setlength{\realevensidemargin}{3cm}  % REAL evenside margin.
\setlength{\hoffset}{0cm}
\setlength{\voffset}{0cm}
%
% Substracción de 1 pulgada de compensación
%  (véase ``Page Layout.png'' para más información)
\addtolength{\realoddsidemargin}{-1in}  % 1 inch = 2.54 cm.
\addtolength{\realevensidemargin}{-1in}
\addtolength{\realtopmargin}{-1in}
%
% Asignación de anchuras y márgenes
% No hay notas al margen
\setlength{\marginparsep}{0cm} % No van a existir notas al margen
\setlength{\marginparwidth}{0cm} % No van a existir notas al margen
%
% Asignación de anchura de texto
\setlength{\textwidth}{15cm}  % Anchura neta del texto (globalmente).
%
% Asignación de márgenes par, impar y en altura
\setlength{\oddsidemargin}{\realoddsidemargin}  % odd-page left margin (global).
\setlength{\evensidemargin}{\realoddsidemargin} % even-page left margin (global).
\setlength{\topmargin}{\realtopmargin}          % top margin (Global).

% Se puede usar también el paquete chngpage.

%%%%%%%%%%%%%%%%%%%%%%%%%%%%%%%%%%%%%%%%%%%%%%%%%%%%%%%%%%%%%%%%%%
%   1 Length commands.          %
%-------------------------------%
% Defines new length command (e.g., \newlength{\gnat}}
% \newlength{}
%
% Set lenght to a value.
% \setlength{\gnat}{length}
% \addtolength{}{}
%
% Sets the value of a length command equal to the width of a specified piece of text; e.g., \settowidth{\parindent}{\em small}.
% \settowidth{}{}
% Set the value of a height. e.g., \settoheight{\parskip}{Gnu}
% \settoheight{}{}
% Set the value that extends below the line. e.g., \settodepth{\parskip}{gnu}.
% \settodepth{}{}
%
% To multiply a length, write: 7.0\gnat = \gnat * 7.0
%%%%%%%%%%%%%%%%%%%%%%%%%%%%%%%%%%%%%%%%%%%%%%%%%%%%%%%%%%%%%%%%%%
