\chapterbeginx{Prólogo}

<<<<<<< HEAD
La presente memoria está dividida en tres partes más apéndices. La primera
parte corresponde con la Introducción, donde presentamos la motivación, objetivos
y fases de realización de este proyecto. La segunda parte constituye el núcleo
del presente PFC y contiene los distintos capítulos en los que se organiza el
libro de prácticas y que constituyen el grueso de este documento. De los 5
capítulos, el primero es introductorio. Los dos siguientes se centran en la
programación de ejecutables en Linux, tratando las estructuras de control en el
capítulo 2 y las subrutinas (funciones) en el capítulo 3. Los dos últimos
capítulos muestran la programación en Bare Metal, explicando las bases en
el capítulo 4 y las interrupciones en el capítulo 5.

Por último la tercera parte contiene las conclusiones del proyecto donde se
discuten las principales aportaciones de este trabajo. En los apéndices hemos
añadido aspectos laterales al PFC pero de suficiente relevancia como para ser
considerados en la memoria, como el apendice A que explica el funcionamiento
de la macro ADDEXC, el apéndice B que muestra todos los detalles de la placa
auxiliar, el apéndice C que nos enseña a agilizar la carga de programas Bare Metal
y por último tenemos el apéndice D, que profundiza en aspectos del GPIO como
las resistencias programables.

A título personal este proyecto no me ha resultado para nada pesado porque
la programación a bajo nivel es un tema que me fascina y la arquitectura ARM
era totalmente nueva para mí. También porque ha requerido dosis altas de
investigación y desarrollo al tratarse de un dispositivo relativamente
reciente, lo cual ha sido gratificante. Espero que este libro de prácticas se
haya impregnado con una parte de ese entusiasmo, y que como alumno curioso
que eres (lo sé porque estás leyéndote el Prólogo) seas capaz de absorver.

=======
%\minitoc
 
>>>>>>> 3d6cfbb4879ae8fab3492658145f40bbd5e4b759
\chapterend
