%%%%%%%%%%%%%%%%%%%%%%%%%%%%%%%%%%%%%%%%%%%%%%%%%%%%%%%%%%%%%%%%%%%
%%% Documento LaTeX                                             %%%
%%%%%%%%%%%%%%%%%%%%%%%%%%%%%%%%%%%%%%%%%%%%%%%%%%%%%%%%%%%%%%%%%%%
% Título: Prólogo
% Autor:  Ignacio Moreno Doblas
% Fecha:  2014-02-01
%%%%%%%%%%%%%%%%%%%%%%%%%%%%%%%%%%%%%%%%%%%%%%%%%%%%%%%%%%%%%%%%%%%

\chapterbeginx{Prólogo}

%\minitoc
  Aquí debe escribirse el prólogo del proyecto fin de carrera.
  \medskip
  
  La calidad en la presentación de los textos y las flexibilidad de \LaTeX\ me llevaron a aprenderlo, a pesar de su difícil curva de aprendizaje.\nli
  Espero que esta plantilla ayude notablemente a suavizar este inconveniente.
  
  Quiero agradecer a las personas que han colaborado en la realización de esta plantilla \LaTeX. Es un sistema muy rápido y cómodo en la generación de este tipo de documentos técnicos y su lectura es francamente agradable.
  
  Animo a todo el mundo a utilizarlo.
  
  Desde la página de la escuela hay disponible también un \miindex{manual de estilo} para ayudar en la redacción y el acabado del proyecto.
  Puede consultarse en~\cite{GuiaEstilo}~\footnote{
    \url{http://www.uma.es/media/files/Manual_de_Estilo_TFG_ETSIT.pdf}
  }.

  También sería interesante hacer dos manuales más: 
\begin{itemize}
  \item{Uno de \LaTeX, que explique con más detalle cómo utilizar este sistema. Aunque en Internet hay muchos disponibles, un manual rápido y directo suavizaría aún más la curva de aprendizaje.\nli
    Quizá, lo más importante es que integre todos los elementos que un usuario necesita, ya que normalmente es necesario acudir a varias fuentes y eso suele requerir demasiado tiempo.\nli
    El capítulo~\ref{chp:ManLaTeX} contiene información orientada a un iniciado en este sistema.}
  
  \item{Y otro, que explique herramientas y métodos útiles que un proyectando puede necesitar en la elaboración del proyecto, tal como llevar un control de versiones de la documentación o el código fuente desarrollado utilizando Assembla\TM. Este último manual es interesante también para muchos jóvenes profesionales, especialmente en el área de desarrollo de sistemas.}
\end{itemize}

  Me reservo el derecho de hacerlo, dado el escaso tiempo del que dispongo.
  \bigskip
  
  Muchas gracias.

\chapterend
