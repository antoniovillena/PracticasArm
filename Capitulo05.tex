\documentclass[12pt,a4paper]{book} % article, report, book.
\input{A1.Preambulo.paquetes.tex}
\input{A2.Preambulo.commandos.tex}
\input{A3.Preambulo.entornos.tex}
\input{A4.Preambulo.estilodepagina.tex}
\usepackage{epstopdf}
\begin{document}

%%%%%%%%%%%%%%%%%%%%%%%%%%%%%%%%%%%%%%%%%%%%%%%%%%%%%%%%%%%%%%

\chapterbegin{Interrupciones hardware}
\label{chp:Interrupciones}
\minitoc

{\bf Objetivos}: En esta sesión vamos a realizar programas que
utilizan dispositivos de E/S haciendo uso del sistema de
interrupciones hardware. Para poder programar los distintos parámetros
que configuran el entorno de las interrupciones es necesario conocer de
forma detallada cómo funcionan los registros asociados, ya que
éste es el mecanismo mediante el cual el procesador se
comunica con los periféricos.

Hacemos incapié en lo de {\it hardware} porque las
{\it interrupciones software} no son más que las llamadas a sistema
que vimos en el capítulo anterior. Ambas comparten vector de interrupciones,
pero las {\it interrupciones software} son más bien llamadas a subrutinas.

\section{Lectura previa}

\subsection{Hardware necesario}

Como vamos a seguir trabajando en {\it Bare Metal} y la complejidad
de los ejemplos va en aumento, no nos queda más remedio que
recurrir a hardware externo a la Raspberry para poder seguir. Se
trata de una placa sencilla, con 2 pulsadores, 6 LEDs y un altavoz
piezoeléctrico.

\begin{longtable}{ p{1.8cm} | p{1.2cm} | p{2cm} | p{5cm}}
\hline
{\bf Nombre} & {\bf GPIO} & {\bf Tipo} & {\bf Descripción} \\ \hline
LED1 & 0 & Salida & Diodo led color rojo \\ \hline
LED2 & 1 & Salida & Diodo led color rojo \\ \hline
LED3 & 17 & Salida & Diodo led color amarillo \\ \hline
LED4 & 22 & Salida & Diodo led color amarillo \\ \hline
LED5 & 10 & Salida & Diodo led color verde \\ \hline
LED6 & 11 & Salida & Diodo led color verde \\ \hline
BOT1 & 21 & Entrada & Pulsador izquierdo \\ \hline
BOT2 & 9 & Entrada & Pulsador derecho \\ \hline
ALT & 4 & Salida & Altavoz piezoeléctrico \\ \hline
\end{longtable}

Es un circuito sencillo y se puede montar en una protoboard sin
problemas, el esquema es el siguiente.

\begin{figure}[h]
  \centering
    \includegraphics[width=14cm]{graphs/circuito.png}
  \caption{Esquema del circuito}
  \label{fig:circuito}
\end{figure}

También se pueden adquirir por internet tarjetas similares, aunque cableadas
a distintos puertos GPIO, por lo que habría que hacer los respectivos cambios
en el código de los ejemplos.

\subsection{Conceptos previos}

Una interrupción es una petición que se hace a la CPU para que detenga el
trabajo que esté realizando y ejecute una rutina determinada. Es uno de los
dos métodos que hay para comunicarse con los periféricos. El otro método,
que ya hemos usado en el capítulo anterior, se llama {\it Polling} y no es
más que una consulta constante sobre el estado de un periférico o un envío
continuo de información.

Existe otro concepto llamado {\it Excepción} que es muy ambiguo. Normalmente
una excepción es una situación de error detectada por la CPU mientras ejecuta
una instrucción, que requiere tratamiento por parte del sistema operativo.
Sin embargo en lo que a arquitectura ARM se refiere, el concepto {\it Excepción}
es diferente. Se trata de un caso general y abarca interrupciones hardware,
interrupciones software y el caso particular de {\it excepción} que explicamos
antes.

Por último tenemos el vector de excepciones, que no es más que una serie de
direcciones a las que la CPU saltará según el tipo de excepción que se produzca.
Lo más normal es que la base del vector de excepciones sea la dirección {\tt 0x00000000},
o sea la dirección más baja de memoria, aunque se puede cambiar dicha dirección base.
La lista del vector de excepciones es la siguiente.

\begin{longtable}{ p{5cm} | p{2cm} | p{4cm}}
\hline
{\bf Excepción} & {\bf Desplaz.} & {\bf Modo} \\ \hline
Reset                   & {\tt 0x00} & Supervisor \\ \hline
Instrucción no definida & {\tt 0x04} & No definido \\ \hline
Interrupción software   & {\tt 0x08} & Supervisor \\ \hline
Error en prefetch       & {\tt 0x0C} & Abort \\ \hline
Error en datos          & {\tt 0x10} & Abort \\ \hline
Reservado               & {\tt 0x14} & Reservado \\ \hline
IRQ                     & {\tt 0x18} & IRQ \\ \hline
FIQ                     & {\tt 0x1C} & FIQ \\ \hline
\end{longtable}

La última columna se refiere al {\it Modo de operación} que comentamos en el primer
capítulo, es un estado en el que se encuentra el procesador con una serie de privilegios
con respecto a otros modos y que gracias a ellos podemos construir un sistema operativo
con diferentes capas. Cada modo tiene sus propios registros {\tt sp} y {\tt lr} de tal
forma que no alteramos la pila de la secuencia de programa que interrumpimos. Incluso
el modo {\tt FIQ} tiene 5 registros generales propios (desde {\tt r8} hasta {\tt r12}),
de esta forma si los empleamos en nuestra rutina de tratamiento no tendremos que
salvaguardarlos en pila.

{\it Reset} es la excepción que se produce al forzar un reinicio en dicha máquina. La
Raspberry no tiene pulsador de reset, pero sí un par de pines (sin montar) para que podamos
añadir un pulsador externo. Este conector se llama {\tt P6} en la Raspberry original y
{\tt RUN} en el modelo B+.

{\it Instrucción no definida} se produce cuando en el flujo de instrucciones nos encontramos
un código de operación que no se corresponde con ninguna instrucción. Normalmente esto se
produce por una corrupción en la memoria de programa o bien que hemos saltado erróneamente
a una zona donde hay datos. También se puede dar el caso de que intentemos ejecutar código
ARM para una plataforma más moderna y nos encontremos con una instrucción no soportada por el
procesador que tenemos.

Las {\it interrupciones software} ya las hemos visto, son las que se invocan con la
instrucción {\tt swi \#n}.

Luego tenemos los errores en {\it prefetch} y {\it datos}. Un error de {\it prefetch} se produce
cuando tratamos de ejecutar una instrucción que momentos antes hemos modificado. Es poco
frecuente y para que se produzca debemos escribir código que sea automodificable, que es
una práctica no deseable y apenas utilizada en dicha arquitectura. Los errores de {\it datos}
son generados normalmente por el manejador de memoria y responden a fallos de alineación, de
traslación, de dominio o de permisos.

Lo siguiente es una entrada reservada, no tiene ninguna funcionalidad ahora
pero es probable que en futuras extensiones sí que la tenga.

Por último están las excepciones que nos interesan y que trataremos en este capítulo, que son
las interrupciones normales {\tt IRQ} y las interrupciones rápidas {\tt FIQ}.

\subsection{Rutina de tratamiento de interrupción}

Es el segmento de código que se ejecuta para atender a una interrupción. Una vez se haya
ejecutado dicha rutina, retomamos la ejecución normal de nuestro programa, justo en la instrucción en
donde lo habíamos interrumpido. Cada rutina de tratamiento debe atender a todas las posibles
fuentes de interrupción de su mismo tipo, con lo que al comienzo de la interrupción se suelen
acceder a los registros asociados para detectar qué periférico ha causado la excepción y
actuar en consecuencia.

Como sólo nos interesan {\tt IRQ} y {\tt FIQ}, a lo sumo sólo tendremos que escribir dos
rutinas de tratamiento distintas. Si se produce una {\tt IRQ}, se ejecutará el código
que se encuentre en la dirección {\tt 0x0018}, mientras que si lo que salta es una {\tt IRQ},
la dirección a ejecutar será {\tt 0x001C}. La diferencia entre una {\tt IRQ} y una
{\tt FIQ} es que esta última tiene sus propios registros desde {\tt r8} hasta {\tt r12}
asociados al modo de operación, con lo que podemos prescindir del salvado y recuperación
de estos registros en la RTI, ahorrando un tiempo que en determinadas aplicaciones de
tiempo real puede ser decisivo.

El esqueleto de una RTI es el siguiente.

\begin{lstlisting}
irq_handler:
        push    {lista registros}
        ...
        pop     {lista registros}
        subs    pc, lr, #4
\end{lstlisting}

Vemos que a diferencia de las subrutinas donde salíamos con {\tt lr}, en una RTI salimos
con {\tt lr-4}, a ello se debe la última instrucción. ¿Y porqué hay un sufijo {\tt s} al final
de la instrucción {\tt sub}? Pues porque se trata de instrucción especial que sirve para
restaurar el registro {\tt cpsr} que había antes de la interrupción.

Cada modo de operación tiene 3 registros replicados: {\tt sp}, {\tt lr} y {\tt spsr}. Cuando
ocurre una interrupción pasamos de modo supervisor a modo IRQ. Los registros {\tt sp}
y {\tt lr} del modo supervisor no se tocan para nada, con lo que tenemos nuestra propia pila
de interrupciones IRQ. El registro {\tt lr} (asociado a IRQ) se carga apuntando a la
instrucción que va después de donde se ha producido la excepción. Además se hace una copia
del registro {\tt cpsr} a otro llamado {\tt spsr} (Saved Program Status Register), porque
dentro de la RTI es probable que queramos modificar {\tt cpsr} y también porque el hecho
de cambiar de modo de operación a IRQ ya ha modificado el registro {\tt cpsr}.

Al terminar la interrupción restauramos {\tt pc} partiendo {\tt lr} y {\tt cpsr} del registro
{\tt spsr}. Al cambiar de modo volvemos al {\tt sp} y {\tt lr} que teníamos en dicho modo,
el resto de registros es responsabilidad nuestra devolverlos al valor que tenían al entrar
en la RTI. Con todo esto conseguimos volver exactamente al punto del que partíamos minimizando las
operaciones que tiene que hacer la RTI y por tanto el retardo asociado. En otras arquitecturas
además de delegar en la RTI este trabajo, se usa la misma pila de programa, lo que puede
ocasionar problemas si nos importa lo que hay debajo de ésta.

\subsection{Inicialización en programa principal}

En realidad la inicialización es lo único que vamos a tener en nuestro programa principal,
puesto que una vez configuradas las interrupciones vamos a controlar el comportamiento del
programa con las mismas. Es decir, nuestro programa principal tras la inicialización será una
instrucción que salta a sí misma continuamente, la misma {\tt infi: b infi} del capítulo
anterior salvo que esta vez no la usamos para dejar al sistema colgado.

El orden recomendado es el siguiente, aunque se pueden cambiar el mismo salvo el último punto.

\begin{itemize}
  \item Escribimos en el vector de excepciones de la dirección de comienzo de la RTI que vayamos
        a usar. En nuestros ejemplos tendremos IRQs y FIQs, por lo que como mucho haremos dos
        escrituras (para dos RTIs distintas).
  \item Inicializamos el puntero de pila (registro {\tt sp}) en todos los modos de operación.
        Tienen que apuntar a zonas distintas dondo sepamos que no habrá conflictos con la
        memoria de programa. En todos los ejemplos que veremos inicializamos la pila de FIQ
        a {\tt 0x4000}, la de IRQ a {\tt 0x8000} y la del modo Supervisor a {\tt 0x8000000}.
  \item Escribimos código de inicialización ajeno al proceso de interrupción, como por ejemplo
        configurar los GPIOs a salidas donde queramos que actúe un LED.
  \item Ahora viene la inicialización de las interrupciones. Aquí le decimos al sistema qué fuentes
        pueden provocar interrupciones, escribiendo en los registros asociados. 
  \item El último paso es habilitar las interrupciones globalmente escribiendo en el registro
        {\tt cpsr}. Lo hacemos indirectamente vía otro registro, y la instrucción tiene otro
        nombre pero hace lo mismo que un {\tt mov}. En concreto se llama {\tt msr}, y también hay
        otra equivalente {\tt mrs} si lo que queremos es leer de {\tt cpsr} a un registro.
  \item Después de esto se acaba la inicialización y tendríamos el bucle infinito del que consta
        nuestro programa principal. Si todo ha ido bien las rutinas de tratamiento de interrupción
        se encargarán de hacer funcionar nuestro programa como queramos.
\end{itemize}






\chapterend{}

%%%%%%%%%%%%%%%%%%%%%%%%%%%%%%%%%%%%%%%%%%%%%%%%%%%%%%%%%%%%%%
\end{document}