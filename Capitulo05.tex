\documentclass[12pt,a4paper]{book} % article, report, book.
\input{A1.Preambulo.paquetes.tex}
\input{A2.Preambulo.commandos.tex}
\input{A3.Preambulo.entornos.tex}
\input{A4.Preambulo.estilodepagina.tex}
\usepackage{epstopdf}
\begin{document}

%%%%%%%%%%%%%%%%%%%%%%%%%%%%%%%%%%%%%%%%%%%%%%%%%%%%%%%%%%%%%%

\chapterbegin{Interrupciones hardware}
\label{chp:Interrupciones}
\minitoc

{\bf Objetivos}: En esta sesión vamos a realizar programas que
utilizan dispositivos de E/S haciendo uso del sistema de
interrupciones hardware. Para poder programar los distintos parámetros
que configuran el entorno de las interrupciones es necesario conocer de
forma detallada cómo funcionan los registros asociados, ya que
éste es el mecanismo mediante el cual el procesador se
comunica con los periféricos.

Hacemos incapié en lo de {\it hardware} porque las
{\it interrupciones software} no son más que las llamadas a sistema
que vimos en el capítulo anterior. Ambas comparten vector de interrupciones,
pero las {\it interrupciones software} son más bien llamadas a subrutinas.

\section{Lectura previa}

\subsection{Hardware necesario}

Como vamos a seguir trabajando en {\it Bare Metal} y la complejidad
de los ejemplos va en aumento, no nos queda más remedio que
recurrir a hardware externo a la Raspberry para poder seguir. Se
trata de una placa sencilla, con 2 pulsadores, 6 LEDs y un altavoz
piezoeléctrico.

\begin{longtable}{ p{1.8cm} | p{1.2cm} | p{2cm} | p{5cm}}
\hline
{\bf Nombre} & {\bf GPIO} & {\bf Tipo} & {\bf Descripción} \\ \hline
LED1 & 0 & Salida & Diodo led color rojo \\ \hline
LED2 & 1 & Salida & Diodo led color rojo \\ \hline
LED3 & 17 & Salida & Diodo led color amarillo \\ \hline
LED4 & 22 & Salida & Diodo led color amarillo \\ \hline
LED5 & 10 & Salida & Diodo led color verde \\ \hline
LED6 & 11 & Salida & Diodo led color verde \\ \hline
BOT1 & 21 & Entrada & Pulsador izquierdo \\ \hline
BOT2 & 9 & Entrada & Pulsador derecho \\ \hline
ALT & 4 & Salida & Altavoz piezoeléctrico \\ \hline
\end{longtable}

Es un circuito sencillo y se puede montar en una protoboard sin
problemas, el esquema es el siguiente.

\begin{figure}[h]
  \centering
    \includegraphics[width=14cm]{graphs/circuito.png}
  \caption{Esquema del circuito}
  \label{fig:circuito}
\end{figure}

También se pueden adquirir por internet tarjetas similares, aunque cableadas
a distintos puertos GPIO, por lo que habría que hacer los respectivos cambios
en el código de los ejemplos.

\subsection{Conceptos previos}

Una interrupción es una petición que se hace a la CPU para que detenga el
trabajo que esté realizando y ejecute una rutina determinada. Es uno de los
dos métodos que hay para comunicarse con los periféricos. El otro método,
que ya hemos usado en el capítulo anterior, se llama {\it Polling} y no es
más que una consulta constante sobre el estado de un periférico o un envío
continuo de información.

Existe otro concepto llamado {\it Excepción} que es muy ambiguo. Normalmente
una excepción es una situación de error detectada por la CPU mientras ejecuta
una instrucción, que requiere tratamiento por parte del sistema operativo.
Sin embargo en lo que a arquitectura ARM se refiere, el concepto {\it Excepción}
es diferente. Se trata de un caso general y abarca interrupciones hardware,
interrupciones software y el caso particular de {\it excepción} que explicamos
antes.

Por último tenemos el vector de excepciones, que no es más que una serie de
direcciones a las que la CPU saltará según el tipo de excepción que se produzca.
Lo más normal es que la base del vector de excepciones sea la dirección {\tt 0x00000000},
o sea la dirección más baja de memoria, aunque se puede cambiar dicha dirección base.
La lista del vector de excepciones es la siguiente.

\begin{longtable}{ p{5cm} | p{2cm} | p{4cm}}
\hline
{\bf Excepción} & {\bf Desplaz.} & {\bf Modo} \\ \hline
Reset                   & {\tt 0x00} & Supervisor \\ \hline
Instrucción no definida & {\tt 0x04} & No definido \\ \hline
Interrupción software   & {\tt 0x08} & Supervisor \\ \hline
Error en prefetch       & {\tt 0x0C} & Abort \\ \hline
Error en datos          & {\tt 0x10} & Abort \\ \hline
Reservado               & {\tt 0x14} & Reservado \\ \hline
IRQ                     & {\tt 0x18} & IRQ \\ \hline
FIQ                     & {\tt 0x1C} & FIQ \\ \hline
\end{longtable}

La última columna se refiere al {\it Modo de operación} que comentamos en el primer
capítulo, es un estado en el que se encuentra el procesador con una serie de privilegios
con respecto a otros modos y que gracias a ellos podemos construir un sistema operativo
con diferentes capas. Cada modo tiene sus propios registros {\tt sp} y {\tt lr} de tal
forma que no alteramos la pila de la secuencia de programa que interrumpimos. Incluso
el modo {\tt FIQ} tiene 5 registros generales propios (desde {\tt r8} hasta {\tt r12}),
de esta forma si los empleamos en nuestra rutina de tratamiento no tendremos que
salvaguardarlos en pila.

{\it Reset} es la excepción que se produce al forzar un reinicio en dicha máquina. La
Raspberry no tiene pulsador de reset, pero sí un par de pines (sin montar) para que podamos
añadir un pulsador externo. Este conector se llama {\tt P6} en la Raspberry original y
{\tt RUN} en el modelo B+.

{\it Instrucción no definida} se produce cuando en el flujo de instrucciones nos encontramos
un código de operación que no se corresponde con ninguna instrucción. Normalmente esto se
produce por una corrupción en la memoria de programa o bien que hemos saltado erróneamente
a una zona donde hay datos. También se puede dar el caso de que intentemos ejecutar código
ARM para una plataforma más moderna y nos encontremos con una instrucción no soportada por el
procesador que tenemos.

Las {\it interrupciones software} ya las hemos visto, son las que se invocan con la
instrucción {\tt swi \#n}.

Luego tenemos los errores en {\it prefetch} y {\it datos}. Un error de {\it prefetch} se produce
cuando tratamos de ejecutar una instrucción que momentos antes hemos modificado. Es poco
frecuente y para que se produzca debemos escribir código que sea automodificable, que es
una práctica no deseable y apenas utilizada en dicha arquitectura. Los errores de {\it datos}
son generados normalmente por el manejador de memoria y responden a fallos de alineación, de
traslación, de dominio o de permisos.

Lo siguiente es una entrada reservada, no tiene ninguna funcionalidad ahora
pero es probable que en futuras extensiones sí que la tenga.

Por último están las excepciones que nos interesan y que trataremos en este capítulo, que son
las interrupciones normales {\tt IRQ} y las interrupciones rápidas {\tt FIQ}.

\subsection{Rutina de tratamiento de interrupción}

Es el segmento de código que se ejecuta para atender a una interrupción. Una vez se ejecute
dicha rutina, retomamos la ejecución normal de nuestro programa, justo en la instrucción en
donde lo habíamos interrumpido. Cada rutina de tratamiento debe atender a todas las posibles
fuentes de interrupción de su mismo tipo, con lo que al comienzo de la interrupción se suelen
acceder a los registros asociados para detectar qué periférico ha causado la excepción y
actuar en consecuencia.

Como sólo nos interesan {\tt IRQ} y {\tt FIQ}, a lo sumo sólo tendremos que escribir dos
rutinas de tratamiento distintas. Si se produce una {\tt IRQ}, se ejecutará el código
que se encuentre en la dirección {\tt 0x0018}, mientras que si lo que salta es una {\tt IRQ},
la dirección a ejecutar será {\tt 0x001C}. Una ventaja que tienen las interrupciones rápidas
{\tt FIQ} es que al estar al final del vector de excepciones no es necesario hacer un salto,
podemos ubicar la rutina de tratamiento de interrupción (en adelante RTI) directamente en
la dirección {\tt 0x001C}. El resto de excepciones requieren un salto para acceder a su
correspondiente RTI. En nuestros ejemplos pondremos un salto también en el caso de {\tt FIQ}
para no tener que reubicar código. Recuerda que nuestro programa Bare Metal comienza en
{\tt 0x8000} por lo que tendríamos que mover nuestro vector de excepciones a la dirección
{\tt 0x0000}.

En el siguiente ejemplo en las etiquetas {\tt lxxxx} las equis hacen referencia a su dirección
final. En las entradadas del vector de excepciones que no utilicemos podemos poner cualquier
cosa que ocupe 4 bytes, nosotros por sencillez pondremos un {\tt NOP}.

\begin{lstlisting}
l0000:  b   reset
l0004:  nop
l0008:  nop
l000C:  nop
l0010:  nop
l0014:  nop
l0018:  b   irqrti
l001C:  b   fiqrti
\end{lstlisting}

El vector de reset podemos hacer que apunte a la primera instrucción Bare Metal, asi de
producirse un Reset se ejecutará de nuevo todo el código. Ahora veamos cómo empezaría nuestro
código inicial, o sea el ubicado en {\tt 0x8000}. Tenemos que 


\chapterend{}

%%%%%%%%%%%%%%%%%%%%%%%%%%%%%%%%%%%%%%%%%%%%%%%%%%%%%%%%%%%%%%
\end{document}