\chapterbeginx{Conclusiones y líneas futuras}

Tras haber estudiado detenidamente las prácticas diseñadas en
torno a la arquitectura x86 hicimos una comparativa entre
arquitecturas. A pesar de ser las enormes diferencias que separan
ARM de x86 hemos tratado de seguir el mismo guión argumental
de las prácticas.

Al comienzo de cada lección suele haber una serie de conceptos
teóricos, esta parte apenas se ha tocado con ligeras modificaciones
para mantener la coherencia con la nueva arquitectura. El problema
viene una vez avanzamos en la lección, ya que la forma de actuar
es completamente distinta. En x86 básicamente tenemos todo el
hardware a nuestra disposición, podemos comunicarnos a muy bajo
nivel mediante puertos E/S, a bajo nivel mediante interrupciones
software a la BIOS, o a medio/bajo haciendo lo mismo bajo MS-DOS.

En ARM está todo más jerarquizado. Desde Linux no podemos
comunicarnos directamente con el hardware, el más bajo nivel al
que aspiramos es con interrupciones software bajo kernel. También
tenemos acceso a librerías, sin embargo estas se hacen con un
mecanismo distinto, mediante llamadas a funciones empleando la
convención AAPCS.

Aunque no sea exactamente igual, hemos creado una equivalencia entre
la BIOS del PC y el kernel de Linux, entre las ISR (Interrupt Service
Routines) del MS-DOS y las llamadas a funciones de librerías externas,
y finalmente hemos recurrido al Bare Metal para mostrar cómo se
accede directamente al hardware.

Hemos estudiado a fondo la plataforma Raspberry Pi, sobre todo lo
referente al subsistema E/S y el acceso en Bare Metal. Hemos
buscado dispositivos que sean simples pero que a la vez abarquen
todos los aspectos requeridos desde un punto de vista didáctico,
sobre todo que puedan ser accedidos tanto mediante polling como
vía interrupciones.

Llegamos a la conclusión de que con el puerto GPIO y el System Timer
se cubrían todos los aspectos, centrando las prácticas Bare Metal
en torno a estos. Vimos la necesidad de diseñar una placa de extensión
para aumentar la funcionalidad de la Raspberry Pi, y de esta forma
ilustrar mejor los ejemplos en Bare Metal. Gracias a los pulsadores,
el buzzer y los LEDs el alumno puede ir experimentando con el código
y aprendiendo los distintos mecanismos que hay para manejarlos.

En la adaptación de las prácticas es donde más cambios hemos
realizado, debido a la diferencia entre arquitecturas y a que una
conversión más literal de las mismas habría resultado un tanto
forzada. Hemos tratado de seguir un orden más natural, por ejemplo
al introducir los LEDs. Primero tratamos de simplemente encender
un LED, luego nos centramos en hacerlo parpadear, y a medida que
las prácticas avanzan hacemos cosas más complicadas como una
secuencia animada que involucre a varios LEDs.

En cuanto al subsistema gráfico hemos experimentado con él, pero al
final hemos decidido excluirlo para las prácticas debido a la
dificultad logística y económica que supone dotar a los laboratorios
de monitores con entrada HDMI ó Video Compuesto. Los que hay sólo
tienen VGA, y aunque hay aparatos conversores, haría falta también
un sistema de conmutación práctico ya que se requiere trabajar a la vez
con el PC y la Raspberry.

Hemos simplificado al máximo el conexionado y el proceso de carga de
programas Bare Metal. En lugar de conectarnos a la Raspberry por el
típico puerto Ethernet para trabajar con Linux, hemos aprovechado el
puerto serie (necesario para Bare Metal) para hacer la misma función,
con lo que nos ahorramos un cable Ethernet y un slot libre en el switch o
router. El único elemento necesario para trabajar con la Raspberry
(aparte de la placa de expansión) es un conversor USB-Serie que cuesta
en torno a un euro. Este dispositivo tendría 3 funciones: sirve para
alimentar la Raspberry, para trabajar con ella en Linux a modo de terminal
y para cargar (también depurar) programas en Bare Metal.

Especial atención hemos puesto en la carga Bare Metal. Aunque en teoría
se puedan ejecutar programas Bare Metal simplemente reemplazando el
archivo kernel.img de la tarjeta SD, a la hora de desarrollar este
mecanismo resulta tedioso. Extraer la tarjeta, meterla en un PC,
reemplazar el archivo, extraerla del PC, introducirla en la Raspberry
y resetear la misma es una tarea que requiere su tiempo. Aunque un minuto
no suponga mucho tiempo si lo haces una sola vez, a medida que vas
probando modificación tras otra, el proceso se vuelve repetitivo y
aburrido. Hemos optimizado el ciclo de desarrollo mediante un bootloader
en la tarjeta SD que lee del puerto serie y ejecuta el programa Bare
Metal que se le envíe. También hemos pedido conversores USB-Serie
especiales que disponen de un pin de más, que empleamos
para forzar el reset de la Raspberry, que de otra forma habría que
desenchufar y enchufar (o a lo sumo pulsar un botón).

De esta forma la carga de programas Bare Metal se vuelve muy ágil, y
en cuestión de 1 ó 2 segundos podemos tener cargado un nuevo programa
desde la última modificación que hagamos.

Hemos tratado de aproximar al alumno a esta plataforma de
la forma más natural posible, empleando la cadena de herramientas GNU
para ensamblado, compilado, enlazado y depuración, que además de ser
gratuita está bastante extendida en la comunidad (e incluida en la
propia distribución). También hemos escogido Raspbian como distribución
de Linux porque es con diferencia la más popular. Está basada en Debian,
por lo que resulta apropiada para usuarios con poca experiencia en Linux.

Nuestra intención es que el alumno además de aprender disfrute de la
asignatura. Con esta plataforma es posible porque si el alumno lo desea
puede adquirir una Raspberry por su propia cuenta y complementar las
prácticas desde casa. Por esta razón hemos dotado al
laboratorio con el doble de placas de expansión que Raspberries. Así
podrán ser prestadas al alumno que las solicite, ya sea para continuar
el trabajo que se ha quedado pendiente en el laboratorio o para
experimentar con sus propios desarrollos.

Como líneas futuras está el ya mencionado subsistema gráfico. También puede
resultar interesante explotar las posibilidades que brinda el coprocesador
matemático (ó VFP) u otros dispositivos más complejos como el teclado,
el audio o el acceso a la tarjeta SD.

Otra línea no explorada puede ser la depuración en Bare Metal, si bien hemos
probado un sistema que permite depurar por el puerto serie, hemos decidido
no incluirlo para no sobrecargar al alumno. Tampoco es realmente necesaria,
los ejemplos Bare Metal son más bien sencillos y el alumno puede repasar
el código para tratar de encontrar dónde falla.

En este proyecto hemos tratado de buscar un acercamiento sencillo y didáctico
para aprender ensamblador y familiarizarse con la plataforma. Pero nada impide
hacer cosas más complejas dando por aprendidos estos conocimientos en
asignaturas posteriores. Un ejemplo puede ser para Sistemas Operativos. Saber
combinar un lenguaje de alto nivel como C con ensamblador y el manejo del
hardware a bajo nivel son una base esencial para esta asignatura.

Por último, los conocimientos adquiridos aquí también pueden resultar
interesantes para proyectos Bare Metal que no sean Sistemas Operativos, hay un
abanico muy amplio: aplicaciones gráficas, sistemas en tiempo real,
sistemas embebidos, etc...

\begin{flushright}
{\large \pfcauthorname}\nli
\today
\end{flushright}
  
\chapterend
