\chapterbeginx{Introducción, Objetivo y Visión general}
\minitoc

\begin{sinopsis}
\label{sec:intro:sinop}
El minicomputador Raspberry Pi es una placa del tamaño de una tarjeta de crédito y un
precio de sólo 30€. El objetivo principal de sus creadores, la Fundación Raspberry Pi,
era promover la enseñanza de conceptos básicos de informática en los colegios e institutos.
Sin embargo, ha terminado convirtiéndose también en un pequeño computador de bajo
coste que se destina a muy diversos usos: servidor multimedia conectado al televisor,
estación base para domótica en el hogar, estaciones meteorológicas, servidor de discos
en red para copias de seguridad, o como un simple ordenador que puede ejecutar
aplicaciones de internet, juegos, ofimática, etc. Esto ha llegado a ser así gracias
a un vertiginoso crecimiento de la comunidad de desarrolladores para Raspberry Pi,
y que estos han explorado casi todas las posibilidades para sacar el máximo partido
de este ordenador de 30€. Esa gran funcionalidad y el bajo coste constituyen el
principal atractivo de esta plataforma para los estudiantes. Sin embargo, para los
docentes del Dept. de Arquitectura de Computadores, la Raspberry Pi ofrece una excusa
perfecta para hacer más amenos y atractivos conceptos a veces complejos, y a veces
también áridos, de asignaturas del área.

Este PFC se enmarca dentro del Proyecto de Innovación Educativa PIE13-082, ``Motivando
al alumno de ingeniería mediante la plataforma Raspberry Pi'' cuyo principal objetivo es
aumentar el grado de motivación del alumno que cursa asignaturas impartidas por el
Departamento de Arquitectura de Computadores. La estrategia propuesta se apoya en el
hecho de que muchos alumnos de Ingeniería perciben que las asignaturas de la carrera
están alejadas de su realidad cotidiana, y que por ello, pierden cierto atractivo. Sin
embargo, bastantes de estos alumnos han comprado o piensan comprar un minicomputador
Raspberry Pi que se caracteriza por proporcionar una gran funcionalidad, gracias a
estar basado en un procesador y Sistema Operativo de referencia en los dispositivos
móviles. En este proyecto proponemos aprovechar el interés que los alumnos ya
demuestran por la plataforma Raspberry Pi, para ponerlo a trabajar en pro del
siguiente objetivo docente: facilitar el estudio de conceptos y técnicas impartidas
en varias asignaturas del Departamento. Cuatro de estas asignaturas son:

\begin{itemize}
  \item{Tecnología de Computadores:} Asignatura obligatoria del módulo de Formación
        Común de las titulaciones de Grado en Ingeniería Informática, Grado en
        Ingeniería de Computadores y Grado en Ingeniería del Software. Es una asignatura
        que se imparte en el primer curso.
  \item{Estructura de Computadores:} Asignatura obligatoria del módulo de Formación
        Común de las titulaciones de Grado en Ingeniería Informática, Grado en
        Ingeniería de Computadores y Grado en Ingeniería del Software. Es una asignatura
        que se imparte en el segundo curso.
  \item{Sistemas Operativos:} Asignatura obligatoria del módulo de Formación Común de las
        titulaciones de Grado en Ingeniería Informática, Grado en Ingeniería de
        Computadores y Grado en Ingeniería del Software. Se imparte en segundo curso.
  \item{Diseño de Sistemas Operativos:} Asignatura obligatoria del módulo de Tecnologías
        Específicas del Grado de Ingeniería de Computadores. Se imparte en tercer curso.
\end{itemize}

En esas cuatro asignaturas, uno de los conceptos más básicos es el de gestión de
interrupciones a bajo nivel. En particular, en Estructura de Computadores, esos
conceptos se ilustraban en el pasado mediante prácticas en PCs con MSDOS y programación
en ensamblador, pero el uso de ese sistema operativo ya no tiene ningún atractivo y
además crea problemas de seguridad en los laboratorios del departamento. Sin embargo,
la plataforma Raspberry Pi se convierte en una herramienta adecuada para trabajar a
nivel de sistema, es económica y ya disponemos de unidades suficientes para usarlas
en los laboratorios (40 equipos para ser exactos).
\end{sinopsis}
 
\sectionx{Objetivo}
\label{sec:intro:obj}
El principal objetivo de este PFC es la creación de un conjunto de prácticas enfocadas
al aprendizaje de la programación en ensamblador, en concreto del ARMv6 que es el
procesador de la plataforma que se va a utilizar para el desarrollo de las prácticas,
así como al manejo a bajo nivel de las interrupciones y la entrada/salida en dicho procesador.

Para alcanzar este objetivo tendremos que abordar las siguientes cuestiones, siempre
bajo la supervisión directa de los tutores, mediante reuniones frecuentes con el alumno
que realiza el PFC:

\begin{enumerate}
  \item Estudio de las prácticas de programación en ensamblador x86 que se vienen realizando
        en la actualidad junto con los objetivos docente perseguidos por las mismas, para
        desarrollar una serie de guiones de prácticas similares, pero adaptadas a las
        peculiaridades de la arquitectura destino (Raspberry Pi). Aparte del enunciado
        de las prácticas, se desarrollará la documentación base necesaria para que el
        alumno pueda desarrollar las mismas.
  \item Estudio del subsistema de entrada/salida de la plataforma Raspberry Pi y su manejo
        a bajo nivel sin el uso de librerías (puertos de entrada/salida, interrupciones,
        temporizadores, salida por pantalla, ...). Se elaborará documentación acerca de todo
        este subsistema, de forma que los alumnos puedan comprender su funcionamiento para
        el posterior desarrollo de las prácticas relacionadas con la entrada/salida. 
  \item Estudio de las prácticas de acceso a puertos de entrada/salida y gestión de
        interrupciones en x86 y desarrollo de prácticas similares adaptadas al subsistema
        de entrada/salida de la Raspberry Pi. Al ser este subsistema bastante distinto
        entre las dos plataformas, habrá que tener especial cuidado en la elaboración del
        guión de prácticas correspondiente, con objeto de que se cumplan los objetivos
        docentes originales y que el futuro alumno de Estructura de Computadores pueda
        llevarlas a cabo de la manera más autónoma posible.
  \item Estudio del subsistema gráfico (GPU) integrado en la Raspberry Pi y su utilización
        desde el lenguaje ensamblador sin el uso de ninguna librería. Elaboración de
        documentación del uso del sistema gráfico desde ensamblador para su uso posterior
        por parte de los alumnos en la realización de las prácticas. 
\end{enumerate}

\sectionx{Metodología y directrices seguidas}
Como se ha presentado en los objetivos de este proyecto, el aspecto principal será el
estudio a bajo nivel del nuevo sistema hardware que se va a utilizar en las prácticas
de laboratorio, para posteriormente, en base a ese conocimiento, desarrollar y
documentar adecuadamente las nuevas prácticas adaptadas a dicho hardware. Para todos
los puntos a estudiar se van a consultar principalmente las abundantes fuentes de
información que existen en internet, lo que supondrá posteriormente una labor
cuidadosa de selección de información y redacción de la documentación adecuada para
el objetivo docente de las prácticas.

El aprendizaje del lenguaje ensamblador del procesador ARM usado en la Raspberry Pi
se puede llevar a cabo leyendo la documentación disponible en \cite{DATH}
y haciendo los tutoriales de \cite{TASM}.
Para la parte más centrada en el hardware como por ejemplo entender el uso del GPIO
(subsistema encargado de la entrada/salida general) para poder controlar los leds
disponibles en la placa. Se consultará la
amplia documentación disponible en internet como por ejemplo los tutoriales disponibles en
\cite{BKPI}
y la descripción los modos de operación de los periféricos conectados al procesador ARM
\cite{ARMP}.

\sectionx{Fases de trabajo y planificación temporal}
Estimamos que el desarrollo completo de este Proyecto Fin de Carrera podría realizarse
a lo largo de unos 6 meses, llevando a cabo las siguientes tareas:

\begin{enumerate}
  \item Estudio básico del lenguaje ensamblador ARM y el tool-chain para compilar de la
        Raspberry Pi necesarios para el posterior desarrollo de todas las prácticas. 
        Estimación temporal: 1 mes.

  \item Estudio de las prácticas de programación en ensamblador existentes en la actualidad,
        adaptación de las mismas al nuevo lenguaje ensamblador y a la nueva plataforma,
        y generación de los nuevos enunciados de las prácticas y la documentación de soporte. 
        Estimación temporal: 2 meses.

  \item Estudio del subsistema de Entrada/Salida de la nueva plataforma para poder desarrollar
        las prácticas con la Entrada/Salida (acceso a puertos, interrupciones, temporizadores
        ...), adaptación de dichas prácticas al nuevo sistema y generación de los nuevos
        enunciados y documentación de soporte. 
        Estimación temporal: 2 meses.

  \item Escritura de la memoria final del PFC. Gran parte de la memoria consistirá en los
        nuevos guiones de prácticas, los cuales se tendrán que confeccionar en LaTeX. La
        mayor parte del contenido de esta memoria habrá sido generado en los puntos anteriores.
        Estimación temporal: 1 mes.
\end{enumerate}

\sectionx{Medios materiales}
Los medios materiales para el desarrollo de este proyecto fin de carrera son los siguientes:
\begin{enumerate}
  \item Plataforma hardware Raspberry Pi, destino de todas las prácticas a desarrollar.
  \item Ordenador personal con conexión a internet para la búsqueda de información.
  \item QEMU, software emulador con soporte ARM para poder realizar tareas sencillas
        directamente en el PC sin tener que usar la Raspberry.
  \item Herramientas de compilación GNU (toolchain) con soporte ARMv6, para la compilación
        del código ensamblador ARM.
  \item Convertidor USB-Serie para comunicarse con la Raspberry.
  \item Placa auxiliar, construida ex profeso para el proyecto empleando un servicio online de
        fabricación de PCBs
\textcolor{blue}{
  \href{http://seeedstudio.com/service/index.php?r=pcb}
  {http://seeedstudio.com/service/index.php?r=pcb}} y comprando los componentes en ebay y aliexpress.
\end{enumerate}

\chapterend
