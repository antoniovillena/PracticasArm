%%%%%%%%%%%%%%%%%%%%%%%%%%%%%%%%%%%%%%%%%%%%%%%%%%%%%%%%%%%%%%%%%%%
%%% Documento LaTeX                                             %%%
%%%%%%%%%%%%%%%%%%%%%%%%%%%%%%%%%%%%%%%%%%%%%%%%%%%%%%%%%%%%%%%%%%%
% Título: Paquetes
% Autor:  Ignacio Moreno Doblas
% Fecha:  2014-02-01
%%%%%%%%%%%%%%%%%%%%%%%%%%%%%%%%%%%%%%%%%%%%%%%%%%%%%%%%%%%%%%%%%%%
% Tabla de materias:
% 1 Codificación e idioma %
% 2 Matemáticas y Física %
% 3 Gráficos%
% 4 Estilo y formato%
%%%%%%%%%%%%%%%%%%%%%%%%%%%%%%%%%%%%%%%%%%%%%%%%%%%%%%%%%%%%%%%%%%%

%1 Codificación e idioma%
\usepackage[utf8]{inputenc} %Codificación en utf8%
\usepackage[spanish]{babel} %Hyphenation (Guionado) en español%
\usepackage[T1]{fontenc} %Codificación de fuente%
%\usepackage{eurofont} %Tipografía euro (€)%
\usepackage{eurosym} %Tipografía euro (€)%
\usepackage{textcomp}

%2 Matemáticas y Física %
% Importante para ecuaciones, magnitudes y unidades%
\usepackage{amssymb,amsmath,latexsym,amsfonts} % paquetes estándar%
\usepackage[squaren]{SIunits} %Paquete para magnitudes y unidades físicas%
\usepackage{ifthen} %sentencias if y while%

%3 Gráficos%
\usepackage{graphics,graphicx} %paquetes gráficos estándar%
\usepackage{wrapfig} %paquete para gráfica lateral%
\usepackage[rflt]{floatflt} %figuras flotantes%
  % \begin{floatingfigure}[r]/[l]{4.5cm}
  % \end{floatingfigure}
\usepackage{graphpap} %comando \graphpaper en el entorno picture%

%4 Estilo y formato%
\usepackage{fancyhdr} %cabeceras y pies mejor que con \pagestyle{}%
\usepackage{titlesec,titletoc} %Formateo de secciones y títulos%
\raggedbottom %Para fragmentar versos en varias páginas%
\usepackage{makeidx} %MakeIndex%
%\usepackage{showidx} % Hace que cada comando \index se imprima en la página donde se ha puesto (útil para corregir los índices)
\usepackage{alltt} % Define el environment alltt, como verbatim, excepto que \, { y } tienen su significado normal. Se describe en el fichero alltt.dtx.
\usepackage[pdftex,bookmarksnumbered,hidelinks]{hyperref} %hyper-references%
\usepackage{minitoc} % Para poner tablas de contenido en cada capítulo.
\usepackage{listings} % Para escribir piezas de código C, Python, etc. %
%listings configuration
\usepackage{color}

\definecolor{mygreen}{rgb}{0,0.6,0}
\definecolor{codegray}{rgb}{0.5,0.5,0.5}
\definecolor{codecyan}{rgb}{0,0.5,0.5}
\definecolor{codegreen}{rgb}{0,0.6,0}
\definecolor{codepurple}{rgb}{0.58,0,0.82}

\lstset{
  language=[ARM]Assembler, %Puede ser C, C++, Java, etc.
  showstringspaces=false,
  formfeed=\newpage,
  tabsize=4,
  commentstyle=\small\color{codegreen},
  basicstyle=\small\ttfamily,
  morekeywords={models, lambda, forms}
  breaklines=true,                 % sets automatic line breaking
  frame=single,                    % adds a frame around the code
  keepspaces=true,                 % keeps spaces in text, useful for keeping indentation of code (possibly needs columns=flexible)
  keywordstyle=\color{blue},       % keyword style
  keywordstyle=[2]\color{codecyan},
}

\usepackage{tipa} % tipografía IPA (International Phonetic Alphabet)
\usepackage{longtable} %Entorno Longtable, fracciona tablas a lo largo de páginas%
\usepackage{colortbl}
\usepackage{acronym}  %Para expandir automáticamente los acrónimos
