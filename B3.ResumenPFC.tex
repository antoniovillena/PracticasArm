%%%%%%%%%%%%%%%%%%%%%%%%%%%%%%%%%%
% Página de resumen del proyecto %
%%%%%%%%%%%%%%%%%%%%%%%%%%%%%%%%%%

\thispagestyle{empty}
\begin{center}
  \Large \sffamily
  Universidad de Málaga\\
  Escuela Técnica Superior de Ingeniería \\
  Informática
\end{center}

\bigskip

\begin{center}
  \Huge\scshape
  \pfctitlename
\end{center}

\bigskip

\begin{center}
  \textbf{REALIZADO POR}\\
  \textsf{\pfcauthorname}
\end{center}

\medskip

\begin{center}
  \textbf{DIRIGIDO POR}\\
  \textsf{\pfctutorname}
\end{center}

\vfill

\begin{minipage}{\textwidth}
\textbf{Dpto. de:} Arquitectura de Computadores (AC)

\medskip

\textbf{Palabras clave:} Raspberry Pi, ensamblador, ARM, prácticas, Bare Metal, interrupciones, bajo nivel

\medskip

\textbf{Titulación:} Ingeniería Informática

\medskip

\textbf{Resumen:}
  Creación de un conjunto de prácticas enfocadas al aprendizaje de la programación en
  ensamblador del ARM, así como al manejo a bajo nivel de las interrupciones y la
  entrada/salida en dicho procesador.

\begin{center} Málaga, \today\end{center}
\end{minipage}

\blankpage
