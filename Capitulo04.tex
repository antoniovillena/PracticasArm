\documentclass[12pt,a4paper]{book} % article, report, book.
%%%%%%%%%%%%%%%%%%%%%%%%%%%%%%%%%%%%%%%%%%%%%%%%%%%%%%%%%%%%%%%%%%%
%%% Documento LaTeX                                             %%%
%%%%%%%%%%%%%%%%%%%%%%%%%%%%%%%%%%%%%%%%%%%%%%%%%%%%%%%%%%%%%%%%%%%
% Título: Paquetes
% Autor:  Ignacio Moreno Doblas
% Fecha:  2014-02-01
%%%%%%%%%%%%%%%%%%%%%%%%%%%%%%%%%%%%%%%%%%%%%%%%%%%%%%%%%%%%%%%%%%%
% Tabla de materias:
% 1 Codificación e idioma %
% 2 Matemáticas y Física %
% 3 Gráficos%
% 4 Estilo y formato%
%%%%%%%%%%%%%%%%%%%%%%%%%%%%%%%%%%%%%%%%%%%%%%%%%%%%%%%%%%%%%%%%%%%

%1 Codificación e idioma%
\usepackage[utf8]{inputenc} %Codificación en utf8%
\usepackage[spanish]{babel} %Hyphenation (Guionado) en español%
\usepackage[T1]{fontenc} %Codificación de fuente%
\usepackage{eurofont} %Tipografía euro (€)%

%2 Matemáticas y Física %
% Importante para ecuaciones, magnitudes y unidades%
\usepackage{amssymb,amsmath,latexsym,amsfonts} % paquetes estándar%
\usepackage[squaren]{SIunits} %Paquete para magnitudes y unidades físicas%
\usepackage{ifthen} %sentencias if y while%

%3 Gráficos%
\usepackage{graphics,graphicx} %paquetes gráficos estándar%
\usepackage{wrapfig} %paquete para gráfica lateral%
\usepackage[rflt]{floatflt} %figuras flotantes%
  % \begin{floatingfigure}[r]/[l]{4.5cm}
  % \end{floatingfigure}
\usepackage{graphpap} %comando \graphpaper en el entorno picture%

%4 Estilo y formato%
\usepackage{fancyhdr} %cabeceras y pies mejor que con \pagestyle{}%
\usepackage{titlesec,titletoc} %Formateo de secciones y títulos%
\raggedbottom %Para fragmentar versos en varias páginas%
\usepackage{makeidx} %MakeIndex%
%\usepackage{showidx} % Hace que cada comando \index se imprima en la página donde se ha puesto (útil para corregir los índices)
\usepackage{alltt} % Define el environment alltt, como verbatim, excepto que \, { y } tienen su significado normal. Se describe en el fichero alltt.dtx.
\usepackage[pdftex,bookmarksnumbered,hidelinks]{hyperref} %hyper-references%
\usepackage{minitoc} % Para poner tablas de contenido en cada capítulo.
\usepackage{listings} % Para escribir piezas de código C, Python, etc. %
%listings configuration
\lstset{
  language=[ARM]Assembler, %Puede ser C, C++, Java, etc.
  showstringspaces=false,
  formfeed=\newpage,
  tabsize=4,
  commentstyle=\itshape,
  basicstyle=\ttfamily,
  morekeywords={models, lambda, forms}
}

\usepackage{tipa} % tipografía IPA (International Phonetic Alphabet)
\usepackage{longtable} %Entorno Longtable, fracciona tablas a lo largo de páginas%
\usepackage{colortbl}
\usepackage{acronym}  %Para expandir automáticamente los acrónimos

%%%%%%%%%%%%%%%%%%%%%%%%%%%%%%%%%%%%%%%%%%%%%%%%%%%%%%%%%%%%%%%%%%%
% Tabla de materias:
% 1 Información del Documento %
% 2 Comandos a nivel de texto %
% 3 Comandos a nivel de entorno %
% 4 Comandos a nivel de página y sección %
% 5 Otros comandos %
%%%%%%%%%%%%%%%%%%%%%%%%%%%%%%%%%%%%%%%%%%%%%%%%%%%%%%%%%%%%%%%%%%%

% 1 Información del Documento %
\newcommand{\pfctitlename}{Guiones de prácticas sobre la plataforma Raspberry Pi}
\newcommand{\pfcauthorname}{Antonio José Villena Godoy}
\newcommand{\pfctutorname}{Rafael Asenjo Plaza \\ Francisco Javier Corbera Peña}
\newcommand{\pfcanno}{2015}

% 2 Comandos a nivel de texto %
\newcommand{\R}{\textsuperscript{\textregistered}}  %Símbolo registrado%
\newcommand{\C}{\textsuperscript{\copyright}} %Símbolo Copyright%
\newcommand{\TM}{\texttrademark} %Símbolo Trade Mark (marca comercial)%

% 2.1 Comandos abreviatura %
\newcommand{\tit}{\textit} %Fuente cursiva (itálica)%
\newcommand{\tbf}{\textbf} %Fuente negrita%
\newcommand{\ttw}[1]{\texttt{#1}} %Fuente máquina de escribir (typewriter)%
%Combinación%
\newcommand{\textittt}[1]{\textit{\texttt{#1}}} %itálica y typewriter%
\newcommand{\textittw}{\textittt} % Otra forma de escribirlo.
\newcommand{\tittw}{\textittw} %Shortened%
\newcommand{\tbftw}[1]{\tbf{\ttw{#1}}}

%Crea una nueva línea y la indenta sin crear interlineado extra.
\newcommand{\nli}{\\ \indent} 

%Para escribir un correo electrónico%
\newcommand{\mailto}[1]{\href{mailto:#1}{#1}}

% Si vas a hacer un uso básico de \index (entradas en el índice de sólo un nivel, sin formatos especiales, etc.), define la orden
\newcommand{\miindex}[1]{#1\index{#1}}

\newcommand{\hs}{\hspace} % Abreviatura espacio horizontal
\newcommand{\vs}{\vspace} % Abreviatura espacio vertical

% Abreviaturas para los conjuntos de números más comunes.
\newcommand{\realnumbers}{\mathbb R}
\newcommand{\naturalnumbers}{\mathbb N}
\newcommand{\integernumbers}{\mathbb Z}
\newcommand{\rationalnumbers}{\mathbb Q}
\newcommand{\complexnumbers}{\mathbb R}
\newcommand{\irrationalnumbers}{\mathbb I}

% Doble barra sobre una letra (para expresar las matrices).
\newcommand{\doublebar}[1]{\bar{\bar{#1}}} 
% Ej: \vector(y) = \doublebar(A) \vector(x) (Stma. lineal de ec.)

% 3 Comandos a nivel de entorno %
\newcommand{\benu}{\begin{enu}} % Begin enumerate
\newcommand{\eenu}{\end{enu}}   % End enumeration

%Comando para escribir código Python
\newcommand{\code}[3]{
  %\hrulefill
  %\subsection*{#1}
  %\subsubsection{#1}
  \lstinputlisting{#2}
  %#1\\
  \begin{table}[h!]
    \centering
    \caption{#1}
    \label{#3}
  \end{table}
  \vspace{2em}
}

% 4 Comandos a nivel de página y sección %
%Crea página en blanco
\newcommand{\blankpage}{\clearpage{\pagestyle{empty}\cleardoublepage}}

% Versión x del comando section: sin numeración pero sí aparece en la tabla de contenidos.
\newcommand{\sectionx}[1]{
  \section*{#1}
  \addcontentsline{toc}{section}{#1}
}

% Versión y del comando section: sin numeración y NO aparece en la tabla de contenidos.
\newcommand{\sectiony}[1]{
  \section*{#1}
}

% Versión x del comando chapter: sin numeración pero sí  aparece en la tabla de contenidos.
\newcommand{\chapterx}[1]{
  \chapter*{#1}
  %\addcontentsline{toc}{chapter}{#1} %Caused by minitoc package%
  \addstarredchapter{#1} %For minitoc package%
}

% substituto del comando \chapter: incluye estilo de página.
\newcommand{\chapterbegin}[1]%
  {%
    \pagestyle{fancy}
    \fancyhead[LE,RO]{\thepage}
    \fancyhead[LO]{Capítulo \thechapter. #1}
    %\fancyhead[RE]{Parte \thepart \rightmark} %
    \fancyhead[RE]{\nouppercase{\rightmark}} %
        
    \chapter{#1}
  }

% Versión x del comando \chapterbegin: sin numeración y aparece en la tabla de contenidos.
\newcommand{\chapterbeginx}[1]%
  {%
    \pagestyle{fancy}
    \fancyhead[RO,LE]{\thepage}
    \fancyhead[RE,LO]{#1}
    %\fancyhead[LO]{Chapter \thechapter}
    %\fancyhead[RE]{Part \thepart} %
    
    \chapterx{#1}
  }

%Fin de capítulo
\newcommand{\chapterend}{\pagestyle{empty}\cleardoublepage \thispagestyle{empty}}
%Si fuera un artículo en lugar un libro, \clearpage en lugar de \cleardoublepage

% 5 Otros comandos %
%\let\Oldpart\part
%\newcommand{\parttitle}{}
%\renewcommand{part}[1]{\Oldpart{#1}\renewcommand{\parttitle}{#1}} %Header customization%

%Cambiar el título índice de capítulo a ``Contenido''.
\renewcommand{\mtctitle}{Contenido}

\dominitoc % Para tablas de contenidos por capítulo.

\addto{\captionsspanish}{
  \renewcommand{\listtablename}{Índice de Tablas}
  \renewcommand{\tablename}{Tabla} } % Por ejemplo, modificar el nombre de 'Cuadro' a 'Tabla'.

\addto{\captionsspanish}{
  \renewcommand{\contentsname}{Índice} }

%Si se desea cambiar el tipo de letra a Arial
% por cualquier razón, descomentar las siguientes
% dos líneas
%\renewcommand{\rmdefault}{phv} % Arial
%\renewcommand{\sfdefault}{phv} % Arial
  
%\addto{\captionsspanish}{
% \renewcommand{\partname}{Fase} }

%\addto{\captionsspanish}{%
%    \renewcommand{\refname}{\vspace{-4.5ex}}} % Para que no aparezca el texto 'referencias' en la bibliografía.

% Modifica el interlineado
%\renewcommand{\baselinestretch}{1.5}

   \definecolor{myfboxbg}{gray}{0.9}
   \newsavebox{\efcaja}
   \newenvironment{myfbox}{\begin{lrbox}{\efcaja}}
               {\end{lrbox}{\colorbox{myfboxbg}{\usebox{\efcaja}}}}

%%%%%%%%%%%%%%%%%%%%%%%%%%%%%%%%%%%%%%%%%%%%%%%%%%%%%%%%%%%%%%%%%%%
% Tabla de materias:
%--------------------%
% 1 dobleindent
% 2 izqindent
% 3 dobleindentx
% 4 ite
% 5 descript
% 6 enu
% 7 itemization
% 8 sinopsis
% 9 objetivo
%%%%%%%%%%%%%%%%%%%%%%%
% Para conocer los parámetros de diseño de las listas, tales como
%  los márgenes izquierdo, derechos y los diferentes saltos,
%  véase el archivo ``List layout.png'' que acompaña esta plantilla.
% Así se conocerá mejor cómo adaptar un entorno según los requisitos 
%  del usuario.

%%%%%%%%%%%%%%%%%%%%%%%
% Definición de longitudes para usar en los entornos:
%
% Normal parskip.
\newlength{\parskipenv}
\setlength{\parskipenv}{\parskip}

\newlength{\parindentenv}
\setlength{\parindentenv}{\parindent}
%%%%%%%%%%%%%%%%%%%%%%%

% 1 dobleindent
%El entorno dobleindent está pensando para escribir párrafos con doble indentación a cada lado.
%Tiene dos parámetros de entrada con las distancias medidas desde los márgenes de página.

\newenvironment{dobleindent}[2]
  %Comienzo de nuevo entorno%
  {
  \begin{list}
    {}
    {
    % Left and right margins:
    \leftmargin = #1 
    \rightmargin = #2
    %
    % Separation from preceding and following text:
    \topsep = 0ex
    \partopsep = 0ex
    \parsep = \parskipenv
    %
    % Indentation for paragraphs:
    \itemsep = \parskipenv
    \itemindent = \parindentenv
    \listparindent = \itemindent
    %
    % Horizontal separation from label:
    \labelsep = 1ex
    \settowidth{\labelwidth}{0cm}
    }
    
     \item}
  % End new env
  {\end{list}}

%%%%%%%%%%%%%%%%%%%%%%%%%%%%%
%2 izqindent
% El entorno izqindent sólo crea un párrafo indentado a la izquierda.
\newenvironment{izqindent}[1]
{
\begin{dobleindent}{#1}{0cm}
}
{
\end{dobleindent}
}

%%%%%%%%%%%%%%%%%%%%%%%%%%%%%
% 3 dobleindentx
% El entorno dobleindentx es una variación del dobleindent usando leftskip y rightskip.
% Aunque es más limitado, también se puede usar.
\newenvironment{dobleindentx}[2] % Sólo funciona en modo paragraph
{ % Preamble
  \leftskip = #1
  \rightskip = #2
}
{ % Postamble
\leftskip = 0cm
\rightskip = 0cm
}

%%%%%%%%%%%%%%%%%%%%%%%%%%%%%
% 4 ite
% El entorno ite es una modificación del entorno itemize estándar de \LaTeX. Puede usarse o modificarse si el usuario lo desea.
% También puede parametrizarse el entorno enumerate o description de forma equivalente.
\newenvironment{ite}
  {
    \begin{izqindent}{\parindent}
    \hspace{-\parindent}  % compensación del sangrado que introduce el entorno.
    \vspace{-1.0\parskip} % compensación del \parskip que introduce el entorno.
    \vspace{-\baselineskip} % compensación por la línea que introduce el entorno.
    \begin{itemize}
  }
  {
    \end{itemize}
    \end{izqindent}
  }

%%%%%%%%%%%%%%%%%%%%%5
% commando stdformat para formatear los entornos descript, enu y itemization.
\newcommand{\stdformat}
  {% Declarations for format presentation.
    %     
    % Separation from preceding and following text:
    \setlength{\topsep}{0ex}%
    \setlength{\partopsep}{0ex} %
    %
    % Horizontal separation from label:
    \labelsep = 1ex
    \setlength{\labelwidth}{0ex}
    %
    % Left and right margins: 
    \setlength{\leftmargin}{1cm}%
    \addtolength{\leftmargin}{\labelsep}
    \setlength{\rightmargin}{0ex}
    %  
    % Indentation for paragraphs:
    \setlength{\itemindent}{-\leftmargin}%
    \addtolength{\itemindent}{1ex}
    \setlength{\listparindent}{\parindent}%
    %   
    % Separation between paragraphs.
    \setlength{\parsep}{\parskipenv}% 
    \setlength{\itemsep}{1ex}
  }

%%%%%%%%%%%%%%%%%%%%%%%%%%%%%
% 5 descript

\newenvironment{descript}
  % Beginning new env def.
  {
    \begin{list}
      {} % No default label for \item.
      {
        % Declarations for format presentation.
        \stdformat
        %
        \renewcommand{\makelabel}[1]{\normalfont\bfseries##1\hfil}
      }
  }
  % Ending new env def.
  {
    \hspace*{\fill} \\ \end{list}
  } % Se introduce un salto de línea para que el texto siguiente esté separado.
%END newenvironment{descript}

%%%%%%%%%%%%%%%%%%%%%%%%%%%%%
% 6 enu
\newcounter{itemnumber} % Counter for the environment.

\newenvironment{enu}
  % Beginning new env def.
  {
    \begin{list}
    {
      \raggedleft \arabic{itemnumber}
    }
    {
      \usecounter{itemnumber}
      \stdformat
    }
  }
  {
    \end{list}
  }

%%%%%%%%%%%%%%%%%%%%%%%%%%%%%
% 7 itemization
\newenvironment{itemization}
  % Beginning new env def.
  {
    \begin{list}
      {$\bullet$} % No default label for \item.
      {
        % Declarations for format presentation.
        \stdformat
      }
  }
  % Ending new env def.
  {
    \end{list}
  }


%%%%%%%%%%%%%%%%%%%%%%%%%%%%%
% 8 sinopsis
\newenvironment{sinopsis}{%[1]{
  \sectiony{Sinopsis}
  %\label{#1}
} {
  \pagebreak
}

%%%%%%%%%%%%%%%%%%%%%%%%%%%%%
% 9 objetivo
\newenvironment{objetivo}{%[1]{
  \sectiony{Objetivo}
  %\label{#1}
} {
}

%%%%%%%%%%%%%%%%%%%%%%%%%%%%%%%%%%%%%%%%%%%%%%%%%%%%%%%%%%%%%%%%%%%
% Tabla de materias:
%--------------------%
% 1 Márgenes de página
%%%%%%%%%%%%%%%%%%%%%%%
% Para conocer los parámetros de diseño de páginas, tales como
%  los márgenes izquierdo, derecho, anchura de página, etc.
%  véase el archivo ``Page layout.png'' que acompaña esta plantilla.
% Así se conocerá mejor cómo adaptar el documento según los 
%  requisitos del usuario.

% 1 Márgenes de página
%-------------------------------%
% Parámetros de estilo de página.
% DIN A4: 29.7 cm x 21 cm
%   área neta: 3 cm + 3 cm + 15 cm.
%
% Definición de márgenes de página
%  even para páginas pares
%  odd  para páginas impares
\newlength{\realoddsidemargin}    % \oddsidemargin menos 1 in.
\newlength{\realevensidemargin}   % \evensidemargin menos 1 in.
\newlength{\realtopmargin}        % \topmargin menos 1 in.
%
% Asignación de márgenes de página
% ASIGNESE en caso de querer cambiarlo
\setlength{\realtopmargin}{2cm}     % REAL top margin.
\setlength{\realoddsidemargin}{3cm}   % REAL oddside margin.
\setlength{\realevensidemargin}{3cm}  % REAL evenside margin.
\setlength{\hoffset}{0cm}
\setlength{\voffset}{0cm}
%
% Substracción de 1 pulgada de compensación
%  (véase ``Page Layout.png'' para más información)
\addtolength{\realoddsidemargin}{-1in}  % 1 inch = 2.54 cm.
\addtolength{\realevensidemargin}{-1in}
\addtolength{\realtopmargin}{-1in}
%
% Asignación de anchuras y márgenes
% No hay notas al margen
\setlength{\marginparsep}{0cm} % No van a existir notas al margen
\setlength{\marginparwidth}{0cm} % No van a existir notas al margen
%
% Asignación de anchura de texto
\setlength{\textwidth}{15cm}  % Anchura neta del texto (globalmente).
%
% Asignación de márgenes par, impar y en altura
\setlength{\oddsidemargin}{\realoddsidemargin}  % odd-page left margin (global).
\setlength{\evensidemargin}{\realoddsidemargin} % even-page left margin (global).
\setlength{\topmargin}{\realtopmargin}          % top margin (Global).

% Se puede usar también el paquete chngpage.

%%%%%%%%%%%%%%%%%%%%%%%%%%%%%%%%%%%%%%%%%%%%%%%%%%%%%%%%%%%%%%%%%%
%   1 Length commands.          %
%-------------------------------%
% Defines new length command (e.g., \newlength{\gnat}}
% \newlength{}
%
% Set lenght to a value.
% \setlength{\gnat}{length}
% \addtolength{}{}
%
% Sets the value of a length command equal to the width of a specified piece of text; e.g., \settowidth{\parindent}{\em small}.
% \settowidth{}{}
% Set the value of a height. e.g., \settoheight{\parskip}{Gnu}
% \settoheight{}{}
% Set the value that extends below the line. e.g., \settodepth{\parskip}{gnu}.
% \settodepth{}{}
%
% To multiply a length, write: 7.0\gnat = \gnat * 7.0
%%%%%%%%%%%%%%%%%%%%%%%%%%%%%%%%%%%%%%%%%%%%%%%%%%%%%%%%%%%%%%%%%%

\begin{document}

\chapterbegin{E/S a bajo nivel}
\label{chp:Subrut}
\minitoc

{\bf Objetivos}: Hasta ahora hemos programado en ensamblador sobre
la capa que nos ofrece el sistema operativo. Nosotros llamamos a una
función y ésta hace todo lo demás: le dice al sistema operativo lo
que tiene que hacer con tal periférico y el sistema operativo (en concreto
el kernel) le envía las órdenes directamente al periférico, al espacio
de memoria donde esté mapeado el mismo.

Lo que vamos a hacer en este capítulo es comunicarnos directamente con los
periféricos, para lo cual debemos prescindir totalmente del sistema operativo.
Este modo de acceder directamente al hardware de la máquina se denomina {\tt Bare Metal},
que traducido viene a ser algo como {\tt Metal desnudo}, haciendo referencia a
que estamos ante la máquina tal y cómo es, sin ninguna capa de abstracción de
por medio.

Veremos ejemplos de acceso directo a periféricos, en concreto el LED y los temporizadores,
que son bastante sencillos de manejar.


\section{Lectura previa}

\subsection{Librerías y Kernel, las dos capas que queremos saltarnos}

Anteriormente hemos utilizado funciones específicas para
comunicarnos con los periféricos. Si por ejemplo necesitamos escribir
en pantalla, llamamos a la función {\tt printf}. Pues bien, entre
la llamada a la función y lo que vemos en pantalla hay 2 capas de por medio.

Una primera capa se encuentra en la librería runtime que acompaña al
ejecutable, la cual incluye sólamente el fragmento de código de la
función que necesitemos, en este caso en {\tt printf}. El resto de
funciones de la librería ({\tt stdio}), si no las invocamos no aparecen
en el ejecutable. El enlazador se encarga de todo esto, tanto de ubicar
las funciones que llamemos desde ensamblador, como de poner la dirección
numérica correcta que corresponda en la instrucción {\tt bl printf}.

Este fragmento de código perteneciente a la primera capa sí que podemos
depurarlo mediante {\tt gdb}. Lo que hace es, a parte del formateo que
realiza la propia función, trasladar al sistema operativo una determinada
cadena para que éste lo muestre por pantalla. Es una especie de traductor
intermedio que nos facilita las cosas. Nosotros desde ensamblador también
podemos hacer llamadas al sistema directamente como veremos posteriormente.

La segunda capa va desde que hacemos la llamada al sistema hasta que se
produce la transferencia de datos al periférico, retornando desde la llamada
al sistema y volviendo a la primera capa, que a su vez retornará el control
a la llamada a librería que hicimos en nuestro programa inicialmente.

En esta segunda capa se ejecuta código del kernel, el cual no podemos depurar.
Además el procesador entra en un modo privilegiado, ya que en modo usuario (el
que se ejecuta en nuestro programa ensamblador y dentro de la librería) no
tenemos privilegios suficientes como para acceder a la zona de memoria que
mapea los periféricos.

Ahora veremos un ejemplo en el cual nos saltamos la capa intermedia para
comunicarnos directamente con el kernel vía llamada al sistema. En este ejemplo
vamos a escribir una simple cadena por pantalla, en concreto "Hola Mundo!".

\begin{lstlisting}[caption={esbn1.c},label={lst:codigoPract4_1},escapeinside={@}{@}]
.data

cadena: .asciz "Hola Mundo!\n"
cadenafin:

.text
.global main
 
main:   push    {r7, lr}           /* preservamos reg.*/
        mov     r0, #1             /* salida estándar */
        ldr     r1, =cadena        /* cadena a enviar */
        mov     r2, #cadenafin-cadena     /* longitud */
        mov     r7, #4             /* seleccionamos la*/
        swi     #0        /* llamada a sistema 'write'*/
        mov     r0, #0             /* devolvemos ok   */
        pop     {r7, lr}           /* recuperamos reg.*/
        bx      lr                 /* salimos de main */
\end{lstlisting}

La instrucción que ejecuta la llamada al sistema es {\tt swi \#0},
siempre tendrá cero como valor inmediato. El código numérico de
la llamada y el número de parámetros podemos buscarlo en cualquier
manual de Linux, buscando "Linux system call table" en Google. En
nuestro caso la llamada {\tt write} se corresponde con el código
4 y acepta tres parámetros: manejador de fichero, dirección de
los datos a escribir (nuestra cadena) y longitud de los datos.

En general se tiende a usar una lista reducida de posibles llamadas
a sistema, y que éstas sean lo más polivalentes posibles. En este
caso vemos que no existe una función específica para escribir en
pantalla. Lo que hacemos es escribir bytes en un fichero, pero usando
un manejador especial conocido como salida estándar, con lo cual todo
lo que escribamos a este fichero especial aparecerá por pantalla.

Pero el propósito de este capítulo no es saltarnos una capa
para comunicarnos directamente con el sistema operativo. Lo que queremos
es saltarnos las dos capas y enviarle órdenes directamente a los periféricos.
Para esto tenemos que saltarnos el sistema operativo, o lo que es lo mismo,
hacer nosotros de sistema operativo para realizar las tareas que queramos.

Este modo de trabajar (como hemos adelantado) se denomina Bare Metal, porque
accedemos a las entrañas del hardware. En él podemos hacer desde cosas
muy sencillas como encender un LED hasta programar desde cero nuestro propio
sistema operativo.

\subsection{Ejecutar código en Bare Metal}

El proceso de arranque de la Raspberry Pi es el siguiente:

\begin{itemize}
  \item Cuando la encendemos, el núcleo ARM está desactivado. Lo primero que se activa es el
        núcleo GPU, que es un procesador totalmente distinto e independiente al ARM. En este
        momento la SDRAM está desactivada.
  \item El procesador GPU empieza a ejecutar la primera etapa del bootloader (son 3 etapas), que
        está almacenada en ROM dentro del mismo chip que comparten ARM y GPU. Esta primera etapa
        accede a la tarjeta SD y lee el fichero {\tt bootcode.bin} en caché L2 y lo ejecuta,
        siendo el código de {\tt bootcode.bin} la segunda etapa del bootloader.
  \item En la segunda etapa se activa la SDRAM y se carga la tercera parte del bootloader, cuyo
        código está repartido entre {\tt loader.bin} (opcional) y {\tt start.elf}.
  \item En tercera y última etapa del bootloader se accede opcionalmente a dos archivos ASCII de
        configuración llamados {\tt config.txt} y {\tt cmdline.txt}. Lo más relevante de esta
        etapa es que cargamos en RAM (en concreto en la dirección 0x8000) el archivo
        {\tt kernel.img} con código ARM, para luego ejecutarlo y acabar con el bootloader, pasando
        el control desde la GPU hacia la CPU.
        Este último archivo es el que nos interesa modificar para nuestros propósitos, ya que es
        lo primero que la CPU ejecuta y lo hace en modo privilegiado, es decir, con acceso total
        al hardware.
\end{itemize}

De todos estos archivos los obligatorios son {\tt bootcode.bin}, {\tt start.elf} y
{\tt kernel.img}. Los dos primeros los bajamos del repositorio oficial
\footnote{\url{https://github.com/raspberrypi}} y el tercero {\tt kernel.img} es el que nosotros
vamos a generar. Estos tres archivos deben estar en el directorio raíz de la primera partición
de la tarjeta SD, la cual debe estar formateada en FAT32.

El proceso completo que debemos repetir cada vez que desarrollemos un programa nuevo
en Bare Metal es el siguiente:

\begin{itemize}
  \item Apagamos la Raspberry.
  \item Extraemos la tarjeta SD.
  \item Introducimos la SD en el lector de nuestro ordenador de desarrollo.
  \item Montamos la unidad y copiamos (sobreescribimos) el kernel.img que acabamos
        de desarrollar.
  \item Desmontamos y extraemos la SD.
  \item Insertamos de nuevo la SD en la Raspberry y la encendemos.
\end{itemize}

Es un proceso sencillo para las prácticas que vamos a hacer, pero para proyectos más largos
se vuelve bastante pesado de realizar. Hay varias alternativas que agilizan el ciclo de
trabajo, donde no es necesario extraer la SD y por tanto podemos actualizar el {\tt kernel.img}
en cuestión de segundos.

\begin{itemize}
  \item Cable JTAG con software Openocd \footnote{\url{http://openocd.sourceforge.net}}
  \item Cable USB-serie desde el ordenador de desarrollo hacia la Raspberry, requiere
        tener instaladas las herramientas de compilación cruzada en el ordenador de desarrollo.
  \item Cable serie-serie que comunica dos Raspberries, una orientada a desarrollo y la otra
        para ejecutar los programas en Bare Metal. No es imprescindible trabajar directamente
        con la Raspberry de desarrollo, podemos acceder vía ssh con nuestro ordenador habitual,
        sin necesidad de tener instaladas las herramientas de compilación en el mismo.
\end{itemize}

En las dos últimas opciones lo que se almacena en el kernel.img de la SD es un bootloader
que lee continuamente del puerto serie y en el momento en que recibe un archivo lo carga
en RAM y lo ejecuta. Para más información sobre estos métodos recomendamos que sigan los
README del siguiente repositorio \footnote{\url{https://github.com/dwelch67/raspberrypi}}.

\subsection{Acceso a periféricos}

Los periféricos se controlan leyendo y escribiendo datos a los registros asociados. No
confundir estos registros con los registros de la CPU. Un registro asociado a un periférico
es un ente, normalmente del mismo tamaño que el ancho del bus de datos, que sirve para
configurar diferentes aspectos del mismo. No se trata de RAM, por lo que no se garantiza que
al leer de un registro obtengamos el último valor que escribimos. Es más, incluso hay
registros que sólo admiten ser leídos y otros que sólo admiten escrituras. La funcionalidad
de los registros también es muy variable, incluso dentro de un mismo registro los diferentes
bits del mismo tienen distinto comportamiento.

Como cada periférico se controla de una forma diferente, no hay más remedio que leerse
el datasheet del mismo si queremos trabajar con él. En nuestro caso queremos encender el LED
de la Raspberry que está conectado al dispositivo GPIO (entrada/salida de propósito general).
Este dispositivo tiene puertos accesibles al exterior mediante un conector de dos filas de
pines en la esquina superior izquierda de la Raspberry, por lo que podemos conseguir los
mismos resultados conectando LEDs sobre diferentes pines de este conector.

El datasheet que tenemos que buscar es el siguiente
\footnote{\url{http://www.raspberrypi.org/wp-content/uploads/2012/02/BCM2835-ARM-Peripherals.pdf}},
ya que el dispositivo GPIO se encuentra en el propio chip (el mismo que contiene CPU y GPU).
En el primer capítulo nos encontramos con el siguiente párrafo.

{\it Physical addresses range from 0×20000000 to 0x20FFFFFF for peripherals. The bus addresses
for peripherals are set up to map onto the peripheral bus address range starting at
0x7E000000. Thus a peripheral advertised here at bus address 0x7Ennnnnn is available
at physical address 0x20nnnnnn.}

Esto quiere decir que los registros de los periféricos están mapeados en memoria en
0×20nnnnnn, que es donde haremos las lecturas y escrituras. Pero debemos
trasladar las direcciones que nos indica el datasheet desde 0x7Ennnnnn hasta 0x20nnnnnn,
porque lo primero son direcciones del bus y lo segundo direcciones físicas, y desde
software empleamos estas últimas.

Para nuestros propósitos nos bastaría con acceder a los registros GPFSELx, GPSETx y GPCLRx,
que se corresponden con 13 de los 41 registros que contiene el GPIO. Vemos la tabla con las
direcciones de estos registros (dos de los cuales no se usan, están reservados)

\begin{longtable}{ p{1.8cm} | p{2cm} | p{5cm} | p{1cm} | p{1cm} }
\hline
{\bf Dirección} & {\bf Nombre} & {\bf Descripción} & {\bf Tam.} & {\bf Tipo} \\ \hline
7E200000 & GPFSEL0 & GPIO Function Select 0 & 32 & R/W \\ \hline
7E200004 & GPFSEL1 & GPIO Function Select 1 & 32 & R/W \\ \hline
7E200008 & GPFSEL2 & GPIO Function Select 2 & 32 & R/W \\ \hline
7E20000C & GPFSEL3 & GPIO Function Select 3 & 32 & R/W \\ \hline
7E200010 & GPFSEL4 & GPIO Function Select 4 & 32 & R/W \\ \hline
7E200014 & GPFSEL5 & GPIO Function Select 5 & 32 & R/W \\ \hline
7E200018 & -       & Reservado              & -  & -   \\ \hline
7E20001C & GPSET0  & GPIO Pin Output Set 0  & 32 & W   \\ \hline
7E200020 & GPSET1  & GPIO Pin Output Set 1  & 32 & W   \\ \hline
7E200024 & -       & Reservado              & -  & -   \\ \hline
7E200028 & GPCLR0  & GPIO Pin Output Clear 0 & 32 & W  \\ \hline
7E20002C & GPCLR1  & GPIO Pin Output Clear 1 & 32 & W  \\ \hline
\end{longtable}

Estos registros controlan 54 pines, que a su vez están agrupados
en 6 grupos funcionales de 10 pines cada uno (excepto el último
que es de 4). El LED interno de la
Raspberry se corresponde con el pin 16 del GPIO, se nombran con
GPIO más el número de pin, en nuestro caso sería GPIO 16. Nótese
que la numeración empieza en 0, desde GPIO 0 hasta GPIO 53. De
todos estos pines sólo diecisiete (0, 1, 4, 7, 8, 9, 10, 11, 14, 15, 17, 18, 22, 23, 24, 25 y 27)
son accesibles a través del conector, y si disponemos del modelo B+ la lista
aumenta en nueve (5, 6, 12, 13, 16, 19, 20, 21 y 26).

Así que la funcionalidad desde GPIO 0 hasta GPIO 9 se controla con
GPFSEL0, desde GPIO 10 hasta GPIO 19 se hace con GPFSEL1 y así
sucesivamente. Nosotros queremos cambiar la funcionalidad de GPIO 16
que está en GPFSEL1. Leyendo más abajo en el datasheet encontramos que
la descripción del registro GPFSEL1 contiene diez grupos funcionales
llamados FSELx (del 10 al 19) de 3 bits cada uno, quedando los dos bits
más altos sin usar. Nos interesa cambiar FSEL16, que serían los bits
20, 19 y 18 de este registro. Las posibles configuraciones son:

\begin{lstlisting}
000 = GPIO Pin 16 is an input
001 = GPIO Pin 16 is an output
100 = GPIO Pin 16 takes alternate function 0
101 = GPIO Pin 16 takes alternate function 1
110 = GPIO Pin 16 takes alternate function 2
111 = GPIO Pin 16 takes alternate function 3
011 = GPIO Pin 16 takes alternate function 4
010 = GPIO Pin 16 takes alternate function 5
\end{lstlisting}

Las funciones alternativas es para dotar a los pines de funciones específicas
para implementar puertos SPI, UART, audio PCM y cosas parecidas. La lista completa
está en la tabla 6-31 (página 102) del datasheet. Nosotros queremos una salida
genérica, nos quedamos con el código {\tt 001}.

Una vez configurado GPIO 16 como salida, ya sólo queda saber cómo poner ceros y unos
en dicho pin, para encender y apagar el LED respectivamente (un cero enciende y un uno
apaga el LED).

Para ello tenemos los registros GPSETn y GPCLRn. En principio parece enrevesado el tener
que usar dos registros distintos para escribir en el puerto GPIO, pero no olvidemos que
para ahorrar recursos varios pines están empaquetados en una palabra de 32 bits. Por lo
que si quisiéramos alterar un único pin tendríamos que leer el registro, modificar el bit
en cuestión sin tocar los demás y escribir el resultado de nuevo en el registro. Por suerte
esto no es necesario con registros separados para setear y resetear, tan sólo necesitamos
una escritura en registro poniendo a 1 los bits que queramos setear/resetear y a 0 los bits
que no queramos modificar.

Leemos el datasheet y comprobamos que es así. Además los 54 pines se reparten entre dos
registros GPSET0/GPCLR0 que contine los 32 primeros y en GPSET1/GPCLR1 están los 22
restantes, quedando libres los 10 bits más significativos de GPSET1/GPCLR1. En nuestro
primer ejemplo de Bare Metal sólo vamos a encender el LED, por lo que emplearemos el
registro GPCLR0.

El ciclo de ensamblado y enlazado también es distinto en un programa Bare Metal. Hasta
ahora hemos creado ejecutables, que tienen una estructura más compleja, con cabecera y
distintas secciones en formato {\tt ELF}. Toda esta información le viene muy bien al
sistema operativo, pero en un entorno Bare Metal no disponemos de él. Lo que se carga
en {\tt kernel.img} es un binario sencillo, sin cabecera, que contiene directamente
el código máquina de nuestro programa y que se cargará en la dirección de RAM {\tt 0x8000}.

Lo que para un ejecutable hacíamos con esta secuencia.
\begin{lstlisting}
        as -o ejemplo.o ejemplo.s
        gcc -o ejemplo ejemplo.o
\end{lstlisting}

En caso de un programa Bare Metal tenemos que cambiarla por esta otra.
\begin{lstlisting}
        as -o ejemplo.o ejemplo.s
        objcopy ejemplo.o -O binary kernel.img
\end{lstlisting}

Otra característica de Bare Metal es que sólo tenemos una sección de código, que no
necesitamos delimitarla con {\tt .text} y tampoco estamos obligados a crear la función
{\tt main}. Al no ejecutar ninguna función no tenemos la posibilidad de salir del
programa con {\tt bx lr}, al fin y al cabo no hay ningún sistema operativo detrás al
que regresar. Nuestro programa debe trabajar en bucle cerrado. En caso de tener una
tarea simple que querramos terminar, es preferible dejar el sistema colgado con un
bucle infinito como última instrucción. Es lo que vamos a hacer en nuestro ejemplo,
que mostramos a continuación.

\begin{lstlisting}[caption={esbn2.c},label={lst:codigoPract4_2},escapeinside={@}{@}]
        .set    GPBASE,   0x20200000
        .set    GPFSEL1,  0x04
        .set    GPSET0,   0x1c
        .set    GPCLR0,   0x28

        ldr     r0, =GPBASE
        ldr     r1, [r0, #GPFSEL1]
        bic     r1, #0b00000000000111000000000000000000
/* guia bits 20..18    xx999888777666555444333222111000*/
        orr     r1, #0b00000000000001000000000000000000
        str     r1, [r0, #GPFSEL1]
/* guia bits 16 a 1    10987654321098765432109876543210*/
        mov     r1, #0b00000000000000010000000000000000
        str     r1, [r0, #GPCLR0]
infi:   b       infi
\end{lstlisting}

El acceso a los registros lo hemos hecho usando la dirección base donde
están mapeados los periféricos {\tt 0x20200000}. Cargamos esta dirección
base en el registro {\tt r0} y codificamos los accesos a los registros
E/S con direccionamiento a memoria empleando distintas constantes como
desplazamiento en función del registro al que queramos acceder.

Observamos que para escribir en 3 bits concretos de {\tt GPFSEL1} primero leemos
el registro en {\tt r1}, luego hacemos un {\tt BIC} (queda más legible que un {\tt AND})
para poner a cero los 3 bits que queremos (del 18 al 20) manteniendo los demás,
más tarde con {\tt ORR} hacemos que el valor de esos 3 bits sea {\tt 001} y finalmente
escribimos de nuevo en {\tt GPFSEL1}. Con esto hemos configurado al pin del LED como salida.
Ya sólo queda encender el LED, enviando un cero con las dos siguientes instrucciones, para
finalmente entrar en un bucle infinito con {\tt infi: b infi}.

\section{Ejemplos de programas Bare Metal}

\subsection{LED parpadeante con bucle de retardo}

La teoría sobre encender y apagar el LED la sabemos. Lo más sencillo que podemos hacer ahora
es hacer que el LED parpadee continuamente. Vamos a intruducir el siguiente programa en la
Raspberry, antes de probarlo piensa un poco cómo se comportaría el siguiente código.

\begin{lstlisting}[caption={esbn3.c},label={lst:codigoPract4_3},escapeinside={@}{@}]
        .set    GPBASE,   0x20200000
        .set    GPFSEL1,  0x04
        .set    GPSET0,   0x1c
        .set    GPCLR0,   0x28

        ldr     r0, =GPBASE
        ldr     r1, [r0, #GPFSEL1]
        bic     r1, #0b00000000000111000000000000000000
        orr     r1, #0b00000000000001000000000000000000
        str     r1, [r0, #GPFSEL1]
bucle:  mov     r1, #0b00000000000000010000000000000000
        str     r1, [r0, #GPSET0]
        mov     r1, #0b00000000000000010000000000000000
        str     r1, [r0, #GPCLR0]
        b       bucle
\end{lstlisting}

Comprobamos que el LED no parpadea sino que está encendido con menos brillo del normal.
En realidad sí que lo hace, sólo que nuestro ojo es demasiado lento como para percibirlo.
Lo siguiente será ajustar la cadencia del parpadeo a un segundo para que podamos observar
el parpadeo. La secuencia sería apagar el LED, esperar medio segundo, encender el LED,
esperar otro medio segundo y repetir el bucle. Sabemos que el procesador de la Raspberry
corre a 700MHz por lo que vamos a suponer que tarde un ciclo de este reloj en ejecutar
cada instrucción. En base a esto vamos a crear dos bucles de retardo uno tras apagar el LED
y otro tras encenderlo de 500ms cada uno. Un bucle de retardo lo único que hace es esperar
tiempo. 

\begin{lstlisting}[caption={Parte de esbn4.c},label={lst:codigoPract4_4},escapeinside={@}{@}]
        ldr     r0, =GPBASE
        ldr     r1, [r0, #GPFSEL1]
        bic     r1, #0b00000000000111000000000000000000
        orr     r1, #0b00000000000001000000000000000000
        str     r1, [r0, #GPFSEL1]
        mov     r1, #0b00000000000000010000000000000000
bucle:  ldr     r2, =350000000/2
ret1:   subs    r2, #1
        bne     ret1
        str     r1, [r0, #GPSET0]
        ldr     r2, =350000000/2
ret2:   subs    r2, #1
        bne     ret2
        str     r1, [r0, #GPCLR0]
        b       bucle
\end{lstlisting}

Observamos que cadencia del LED es demasiado lenta algo así como 3 segundos, lo que quiere
decir que cada iteración del bucle de retardo tarda más de los dos ciclos que hemos supuesto.
Probamos con cronómetro en mano distintos valores del divisor de {\tt 350000000} hasta dar con
el valor 6, aún así no acaba de dar una cadencia exacta de un segundo. Hemos comprobado
empíricamente que las instrucciones {\tt subs} y {\tt bne} tardan en conjunto unos 6 ciclos
por cada iteración. Esto se debe a una dependencia de datos (ya que el flag que altera
la orden {\tt subs} es requerido justo después por la instrucción {\tt bne}) y que los saltos
condicionales suelen ser lentos.

\subsection{LED parpadeante con temporizador interno}

Viendo lo poco preciso que es temporizar con el bucle de retardo, buscamos otra alternativa.
Lo más sencillo que encontramos es el {\tt ARM Timer} del capítulo 14 del datasheet.

El temporizador lo configuramos en {\tt Free Running} (cuenta hacia delante sin parar) con el
preescaler a 0xF9 (249) ya que la frecuencia se divide por el valor del preescaler más uno. Como
el reloj de sistema va a 250Mhz y al pasar por el preescaler se divide entre 250, la frecuencia
a la que se incrementa el valor del temporizador es de 1MHz. Para temporizar medio segundo lo
único que tenemos que hacer es esperar a que el contador se incremente en medio millón. El código
final quedaría así.

\begin{lstlisting}[caption={esbn5.c},label={lst:codigoPract4_5},escapeinside={@}{@}]
        .set    GPBASE,   0x20200000
        .set    GPFSEL1,        0x04
        .set    GPSET0,         0x1c
        .set    GPCLR0,         0x28
        .set    TIMERBASE,0x2000B400
        .set    TIMER_CTL,      0x08
        .set    TIMER_CNT,      0x20

        ldr     r0, =GPBASE
        ldr     r1, [r0, #GPFSEL1]
        bic     r1, #0b00000000000111000000000000000000
        orr     r1, #0b00000000000001000000000000000000
        str     r1, [r0, #GPFSEL1]
        mov     r1, #0b00000000000000010000000000000000
        ldr     r2, =TIMERBASE
        ldr     r3, =0x00f90200
        str     r3, [r2, #TIMER_CTL]
bucle:  bl      espera
        str     r1, [r0, #GPSET0]
        bl      espera
        str     r1, [r0, #GPCLR0]
        b       bucle

/* rutina que espera medio segundo */
espera: ldr     r3, [r2, #TIMER_CNT]
        ldr     r4, =500000
        add     r4, r3
ret1:   ldr     r3, [r2, #TIMER_CNT]
        cmp     r3, r4
        bne     ret1
        bx      lr
\end{lstlisting}



\chapterend{}

%%%%%%%%%%%%%%%%%%%%%%%%%%%%%%%%%%%%%%%%%%%%%%%%%%%%%%%%%%%%%%
\end{document}