\chapterbegin{E/S a bajo nivel}
\label{chp:Subrut}
\minitoc

{\bf Objetivos}: Hasta ahora hemos programado en ensamblador sobre
la capa que nos ofrece el sistema operativo. Nosotros llamamos a una
función y ésta hace todo lo demás: le dice al sistema operativo lo
que tiene que hacer con tal periférico y el sistema operativo (en concreto
el kernel) le envía las órdenes directamente al periférico, al espacio
de memoria donde esté mapeado el mismo.

Lo que vamos a hacer en este capítulo es comunicarnos directamente con los
periféricos, para lo cual debemos prescindir totalmente del sistema operativo.
Este modo de acceder directamente al hardware de la máquina se denomina {\tt Bare Metal},
que traducido viene a ser algo como {\tt Metal desnudo}, haciendo referencia a
que estamos ante la máquina tal y cómo es, sin ninguna capa de abstracción de
por medio.

Veremos ejemplos de acceso directo a periféricos, en concreto al LED de la placa
auxiliar (ver apéndice \ref{chp:PlacaAux}) y a los temporizadores, que son
bastante sencillos de manejar.

\section{Lectura previa}

\subsection{Librerías y Kernel, las dos capas que queremos saltarnos}

Anteriormente hemos utilizado funciones específicas para
comunicarnos con los periféricos. Si por ejemplo necesitamos escribir
en pantalla, llamamos a la función {\tt printf}. Pues bien, entre
la llamada a la función y lo que vemos en pantalla hay 2 capas software
de por medio.

Una primera capa se encuentra en la librería runtime que acompaña al
ejecutable, la cual incluye sólamente el fragmento de código de la
función que necesitemos, en este caso en {\tt printf}. El resto de
funciones de la librería ({\tt stdio}), si no las invocamos no aparecen
en el ejecutable. El enlazador se encarga de todo esto, tanto de ubicar
las funciones que llamemos desde ensamblador, como de poner la dirección
numérica correcta que corresponda en la instrucción {\tt bl printf}.

Este fragmento de código perteneciente a la primera capa sí que podemos
depurarlo mediante {\tt gdb}. Lo que hace es, a parte del formateo que
realiza la propia función, trasladar al sistema operativo una determinada
cadena para que éste lo muestre por pantalla. Es una especie de traductor
intermedio que nos facilita las cosas. Nosotros desde ensamblador también
podemos hacer llamadas al sistema directamente como veremos posteriormente.

La segunda capa va desde que hacemos la llamada al sistema (System Call o Syscall) hasta que se
produce la transferencia de datos al periférico, retornando desde la llamada
al sistema y volviendo a la primera capa, que a su vez retornará el control
a la llamada a librería que hicimos en nuestro programa inicialmente.

En esta segunda capa se ejecuta código del kernel, el cual no podemos depurar.
Además el procesador entra en un modo privilegiado, ya que en modo usuario (el
que se ejecuta en nuestro programa ensamblador y dentro de la librería) no
tenemos privilegios suficientes como para acceder a la zona de memoria que
mapea los periféricos.

La función {\tt printf} es una función de la librería del lenguaje C. Como vemos en
la figura, esta función internamente llama a la System Call (rutina del Kernel del SO)
{\tt write} que es la que se ejecuta en modo supervisor y termina accediendo a
los periféricos (en este caso al terminal o pantalla donde aparece el mensaje). En la
figura \ref{fig:capas} podemos ver el código llamador junto con las dos capas.

\begin{figure}[h]
  \centering
    \includegraphics[width=14cm]{graphs/capas.png}
  \caption{Funcionamiento de una llamada a printf}
  \label{fig:capas}
\end{figure}

Ahora veremos un ejemplo en el cual nos saltamos la capa intermedia para
comunicarnos directamente con el kernel vía llamada al sistema. En este ejemplo
vamos a escribir una simple cadena por pantalla, en concreto "Hola Mundo!".

\begin{lstlisting}[caption={esbn1.s},label={lst:codigoPract4_1},escapeinside={@}{@}]
.data

cadena: .asciz "Hola Mundo!\n"
cadenafin:

.text
.global main
 
main:   push    {r7, lr}           /* preservamos reg.*/
        mov     r0, #1             /* salida estándar */
        ldr     r1, =cadena        /* cadena a enviar */
        mov     r2, #cadenafin-cadena     /* longitud */
        mov     r7, #4             /* seleccionamos la*/
        swi     #0        /* llamada a sistema 'write'*/
        mov     r0, #0             /* devolvemos ok   */
        pop     {r7, lr}           /* recuperamos reg.*/
        bx      lr                 /* salimos de main */
\end{lstlisting}

La instrucción que ejecuta la llamada al sistema es {\tt swi \#0},
siempre tendrá cero como valor inmediato. El código numérico de
la llamada y el número de parámetros podemos buscarlo en cualquier
manual de Linux, buscando ``Linux system call table'' en Google. En
nuestro caso la llamada {\tt write} se corresponde con el código
4 y acepta tres parámetros: manejador de fichero, dirección de
los datos a escribir (nuestra cadena) y longitud de los datos. En
nuestro ejemplo, el manejador de fichero es el {\tt 1}, que está
conectado con la salida estándar o lo que es lo mismo, con la pantalla.

En general se tiende a usar una lista reducida de posibles llamadas
a sistema, y que éstas sean lo más polivalentes posibles. En este
caso vemos que no existe una función específica para escribir en
pantalla. Lo que hacemos es escribir bytes en un fichero, pero usando
un manejador especial conocido como salida estándar, con lo cual todo
lo que escribamos a este fichero especial aparecerá por pantalla.

Pero el propósito de este capítulo no es saltarnos una capa
para comunicarnos directamente con el sistema operativo. Lo que queremos
es saltarnos las dos capas y enviarle órdenes directamente a los periféricos.
Para esto tenemos prescindir del sistema operativo, o lo que es lo mismo,
hacer nosotros de sistema operativo para realizar las tareas que queramos.

Este modo de trabajar (como hemos adelantado) se denomina Bare Metal, porque
accedemos a las entrañas del hardware. En él podemos hacer desde cosas
muy sencillas como encender un LED hasta programar desde cero nuestro propio
sistema operativo.

\subsection{Ejecutar código en Bare Metal}
\label{sec:Ejecutar}

El ciclo de ensamblado y enlazado es distinto en un programa Bare Metal. Hasta
ahora hemos creado ejecutables, que tienen una estructura más compleja, con cabecera y
distintas secciones en formato {\tt ELF} \cite{FELF}. Toda esta información le viene muy bien al
sistema operativo, pero en un entorno Bare Metal no disponemos de él. Lo que se carga
en {\tt kernel.img} es un binario sencillo, sin cabecera, que contiene directamente
el código máquina de nuestro programa y que se cargará en la dirección de RAM {\tt 0x8000}.

Lo que para un ejecutable hacíamos con esta secuencia.
\begin{lstlisting}
        as -o ejemplo.o ejemplo.s
        gcc -o ejemplo ejemplo.o
\end{lstlisting}

En caso de un programa Bare Metal tenemos que cambiarla por esta otra.
\begin{lstlisting}
        as -o ejemplo.o ejemplo.s
        ld -e 0 -Ttext=0x8000 -o ejemplo.elf ejemplo.o
        objcopy ejemplo.elf -O binary kernel.img
\end{lstlisting}

Otra característica de Bare Metal es que sólo tenemos una sección de código (la sección
.text), y no estamos obligados a crear la función
{\tt main}. Al no ejecutar ninguna función no tenemos la posibilidad de salir del
programa con {\tt bx lr}, al fin y al cabo no hay ningún sistema operativo detrás al
que regresar. Nuestro programa debe trabajar en bucle cerrado. En caso de tener una
tarea simple que queramos terminar, es preferible dejar el sistema colgado con un
bucle infinito como última instrucción.

El proceso de arranque de la Raspberry Pi es el siguiente:

\begin{itemize}
  \item Cuando la encendemos, el núcleo ARM está desactivado. Lo primero que se activa es el
        núcleo GPU, que es un procesador totalmente distinto e independiente al ARM. En este
        momento la SDRAM está desactivada.
  \item El procesador GPU empieza a ejecutar la primera etapa del bootloader (son 3 etapas), que
        está almacenada en ROM dentro del mismo chip que comparten ARM y GPU. Esta primera etapa
        accede a la tarjeta SD y lee el fichero {\tt bootcode.bin} en caché L2 y lo ejecuta,
        siendo el código de {\tt bootcode.bin} la segunda etapa del bootloader.
  \item En la segunda etapa se activa la SDRAM y se carga la tercera parte del bootloader, cuyo
        código está repartido entre {\tt loader.bin} (opcional) y {\tt start.elf}.
  \item En tercera y última etapa del bootloader se accede opcionalmente a dos archivos ASCII de
        configuración llamados {\tt config.txt} y {\tt cmdline.txt}. Lo más relevante de esta
        etapa es que cargamos en RAM (en concreto en la dirección 0x8000) el archivo
        {\tt kernel.img} con código ARM, para luego ejecutarlo y acabar con el bootloader, pasando
        el control desde la GPU hacia la CPU.
        Este último archivo es el que nos interesa modificar para nuestros propósitos, ya que es
        lo primero que la CPU ejecuta y lo hace en modo privilegiado, es decir, con acceso total
        al hardware.
\end{itemize}

De todos estos archivos los obligatorios son {\tt bootcode.bin}, {\tt start.elf} y
{\tt kernel.img}. Los dos primeros los bajamos del repositorio oficial
\textcolor{blue}{
  \href{https://github.com/raspberrypi}
  {https://github.com/raspberrypi}}
y el tercero {\tt kernel.img} es el que nosotros
vamos a generar. Estos tres archivos deben estar en el directorio raíz de la primera partición
de la tarjeta SD, la cual debe estar formateada en FAT32.

El proceso completo que debemos repetir cada vez que desarrollemos un programa nuevo
en Bare Metal es el siguiente:

\begin{itemize}
  \item Apagamos la Raspberry.
  \item Extraemos la tarjeta SD.
  \item Introducimos la SD en el lector de nuestro ordenador de desarrollo.
  \item Montamos la unidad y copiamos (sobreescribimos) el kernel.img que acabamos
        de desarrollar.
  \item Desmontamos y extraemos la SD.
  \item Insertamos de nuevo la SD en la Raspberry y la encendemos.
\end{itemize}

Es un proceso sencillo para las prácticas que vamos a hacer, pero para proyectos más largos
se vuelve bastante tedioso. Hay varias alternativas que agilizan el ciclo de
trabajo, donde no es necesario extraer la SD y por tanto podemos actualizar el {\tt kernel.img}
en cuestión de segundos. Estas alternativas son:

\begin{itemize}
  \item Cable JTAG con software Openocd:
\textcolor{blue}{
  \href{http://openocd.sourceforge.net}
  {http://openocd.sourceforge.net}}
  \item Cable USB-serie desde el ordenador de desarrollo hacia la Raspberry, requiere
        tener instaladas las herramientas de compilación cruzada en el ordenador de desarrollo.
  \item Cable serie-serie que comunica dos Raspberries, una orientada a desarrollo y la otra
        para ejecutar los programas en Bare Metal. No es imprescindible trabajar directamente
        con la Raspberry de desarrollo, podemos acceder vía ssh con nuestro ordenador habitual,
        sin necesidad de tener instaladas las herramientas de compilación en el mismo.
\end{itemize}

Las dos últimas opciones están detalladas en el apéndice \ref{chp:SerieBoot}.
Básicamente se trata de meter en el kernel.img de la SD un programa especial (llamado
bootloader) que lee continuamente del puerto serie y en el momento en que recibe
un archivo del tipo {\tt kernel.img}, lo carga en RAM y lo ejecuta.

\section{Acceso a periféricos}

Los periféricos se controlan leyendo y escribiendo datos a los registros asociados o puertos
de E/S. No
confundir estos registros con los registros de la CPU. Un puerto asociado a un periférico
es un ente, normalmente del mismo tamaño que el ancho del bus de datos, que sirve para
configurar diferentes aspectos del mismo. No se trata de RAM, por lo que no se garantiza que
al leer de un puerto obtengamos el último valor que escribimos. Es más, incluso hay
puertos que sólo admiten ser leídos y otros que sólo admiten escrituras. La funcionalidad
de los puertos también es muy variable, incluso dentro de un mismo puerto los diferentes
bits del mismo tienen distinto comportamiento.

Como cada periférico se controla de una forma diferente, no hay más remedio que leerse
el datasheet del mismo si queremos trabajar con él. De ahora en adelante usaremos una placa
auxiliar, descrita en el apéndice \ref{chp:PlacaAux}, y que conectaremos a la fila inferior
del conector GPIO según la figura \ref{fig:posicionaux}. En esta sección explicaremos cómo
encender un LED de esta placa auxiliar.

\begin{figure}[h]
  \centering
    \includegraphics[width=14cm]{graphs/posicionaux.jpg}
  \caption{Colocación de la placa auxiliar}
  \label{fig:posicionaux}
\end{figure}


\subsection{GPIO (General-Purpose Input/Output)}

El GPIO es un conjunto de señales mediante las cuales la CPU se comunica con distintas partes
de la Rasberry tanto internamente (audio analógico, tarjeta SD o LEDs internos) como
externamente a través de los conectores P1 y P5. Como la mayor parte de las señales se
encuentran en el conector P1 (ver figura \ref{fig:posiciongpio}), normalmente este conector se denomina GPIO. Nosotros
no vamos a trabajar con señales GPIO que no pertenezcan a dicho conector, por lo que no
habrá confusiones.

\begin{figure}[h]
  \centering
    \includegraphics[width=14cm]{graphs/posiciongpio.jpg}
  \caption{Posición del puerto GPIO}
  \label{fig:posiciongpio}
\end{figure}

El GPIO contiene en total 54 señales, de las cuales 17 están disponibles a través del conector
GPIO (26 en los modelos A+/B+). Como nuestra placa auxiliar emplea la fila inferior
del conector, sólo dispondremos de 9 señales.

Los puertos del GPIO están mapeados en memoria, tomando como base la dirección 0x20200000.
Para nuestros propósitos de esta lección nos basta con acceder a los puertos GPFSELn,
GPSETn y GPCLRn. A continuación tenemos la tabla con las direcciones de estos puertos.

\begin{table}
\centering
\begin{tabular}{ p{1.8cm} | p{2cm} | p{5cm} | p{1cm} }
{\bf Dirección} & {\bf Nombre} & {\bf Descripción} & {\bf Tipo} \\
\hline
20200000 & GPFSEL0 & Selector de función 0 & R/W \\
20200004 & GPFSEL1 & Selector de función 1 & R/W \\
20200008 & GPFSEL2 & Selector de función 2 & R/W \\
2020000C & GPFSEL3 & Selector de función 3 & R/W \\
20200010 & GPFSEL4 & Selector de función 4 & R/W \\
20200014 & GPFSEL5 & Selector de función 5 & R/W \\
2020001C & GPSET0  & Pin a nivel alto 0 & W   \\
20200020 & GPSET1  & Pin a nivel alto 1 & W   \\
20200028 & GPCLR0  & Pin a nivel bajo 0 & W  \\
2020002C & GPCLR1  & Pin a nivel bajo 1 & W  \\
\end{tabular}
\end{table}

\subsubsection{GPFSELn}

Las 54 señales/pines las separamos en 6
grupos funcionales de 10 señales/pines cada uno (excepto el último
que es de 4) para programarlas mediante GPFSELn.

El LED que queremos controlar se corresponde con la señal número 9 del puerto GPIO.
Se nombran con GPIO más el número correspondiente, en nuestro caso sería GPIO 9. Nótese
que la numeración empieza en 0, desde GPIO 0 hasta GPIO 53.

Así que la funcionalidad desde GPIO 0 hasta GPIO 9 se controla con
GPFSEL0, desde GPIO 10 hasta GPIO 19 se hace con GPFSEL1 y así
sucesivamente. Nosotros queremos encender el primer LED rojo de la placa auxiliar.
En la figura \ref{fig:pinout} vemos que el primer LED rojo se corresponde con GPIO 9.
Para cambiar la funcionalidad de GPIO 9 nos toca actuar sobre GPFSEL0. Por
defecto cuando arranca la Raspberry todos los pines están preconfigurados
como entradas, con lo que los LEDs de nuestra placa auxiliar
están apagados. Es más, aunque lo configuremos como salida, tras el reset,
los pines se inicializan al valor cero (nivel bajo), por lo que podemos presuponer que
todos los LEDs estarán apagados, incluso después de programarlos como salidas.

El puerto GPFSEL0 contiene diez grupos funcionales
llamados FSELx (del 0 al 9) de 3 bits cada uno, quedando los dos bits
más altos sin usar. Nos interesa cambiar FSEL9, que sería el que se corresponde
con el primer LED rojo, el que queremos encender. Las posibles
configuraciones para cada grupo son:

\begin{lstlisting}
000 = GPIO Pin X es una entrada
001 = GPIO Pin X es una salida
100 = GPIO Pin X toma función alternativa 0
101 = GPIO Pin X toma función alternativa 1
110 = GPIO Pin X toma función alternativa 2
111 = GPIO Pin X toma función alternativa 3
011 = GPIO Pin X toma función alternativa 4
010 = GPIO Pin X toma función alternativa 5
\end{lstlisting}

Las funciones alternativas son para dotar a los pines de funcionalidad específicas
como puertos SPI, UART, audio PCM y cosas parecidas. La lista completa
está en la tabla 6-31 (página 102) del datasheet \cite{ARMP}. Nosotros queremos una salida
genérica, así que nos quedamos con el código {\tt 001} para el grupo funcional FSEL9 del
puerto {\tt GPFSEL0} que es el que corresponde al GPIO 9.


\subsubsection{GPSETn y GPCLRn}

Los 54 pines se reparten entre dos
puertos GPSET0/GPCLR0, que contienen los 32 primeros, y en GPSET1/GPCLR1 están los 22
restantes, quedando libres los 10 bits más significativos de GPSET1/GPCLR1.

Una vez configurado GPIO 9 como salida, ya sólo queda saber cómo poner un cero o un
uno en la señal GPIO 9, para apagar y encender el primer LED de la placa auxiliar
respectivamente (un cero apaga y un uno enciende el LED).

Para ello tenemos los puertos GPSETn y GPCLRn, donde GPSETn pone un 1 y GPCLRn pone
un 0. En principio parece enrevesado el tener
que usar dos puertos distintos para escribir en el puerto GPIO, pero no olvidemos que
para ahorrar recursos varios pines están empaquetados en una palabra de 32 bits. Si
sólo tuviéramos un puerto y quisiéramos alterar un único pin tendríamos que leer el
puerto, modificar el bit en cuestión sin tocar los demás y escribir el resultado de
nuevo en el puerto. Por suerte esto no es necesario con puertos separados para setear
y resetear, tan sólo necesitamos una escritura en puerto poniendo a 1 los bits que
queramos setear/resetear y a 0 los bits que no queramos modificar.

En la figura \ref{fig:pinoutpeque} vemos cómo está hecho el conexionado de la placa auxiliar.

\begin{figure}[h]
  \centering
    \includegraphics[width=10cm]{graphs/RaspberryGPIOaux.png}
  \caption{Correspondencia LEDs y GPIO}
  \label{fig:pinoutpeque}
\end{figure}


En nuestro primer ejemplo de Bare Metal sólo vamos a encender el primer LED rojo
de la placa auxiliar, que como hemos dicho se corresponde con el GPIO 9 así que
tendremos que actuar sobre el bit 9 del registro {\tt GPSET0}.

Resumiendo, los puertos a los que accedemos para encender y apagar el LED
vienen indicados en la figura \ref{fig:gpio1}.

\begin{figure}[h]
  \centering
    \includegraphics[width=14cm]{graphs/gpio1.png}
  \caption{Puertos LED}
  \label{fig:gpio1}
\end{figure}

El siguiente código (listado \ref{lst:codigoPract4_2}) muestra cómo hemos de proceder.

\begin{lstlisting}[caption={esbn2.s},label={lst:codigoPract4_2}]
        .set    GPBASE,   0x20200000
        .set    GPFSEL0,  0x00
        .set    GPSET0,   0x1c
.text
        ldr     r0, =GPBASE
/* guia bits           xx999888777666555444333222111000*/
        mov     r1, #0b00001000000000000000000000000000
        str     r1, [r0, #GPFSEL0]  @ Configura GPIO 9
/* guia bits           10987654321098765432109876543210*/
        mov     r1, #0b00000000000000000000001000000000
        str     r1, [r0, #GPSET0]   @ Enciende GPIO 9
infi:   b       infi
\end{lstlisting}

El acceso a los puertos lo hemos hecho usando la dirección base donde
están mapeados los periféricos {\tt 0x20200000}. Cargamos esta dirección
base en el registro {\tt r0} y codificamos los accesos a los puertos
E/S con direccionamiento a memoria empleando distintas constantes como
desplazamiento en función del puerto al que queramos acceder.

El código simplemente escribe dos constantes en dos puertos: {\tt GPFSEL0} y {\tt GPSET0}.
Con la primera escritura configuramos el LED como salida y con la segunda escritura lo
encendemos, para finalmente entrar en un bucle infinito con {\tt infi: b infi}.

\subsubsection{Otros puertos}

Ya hemos explicado los puertos que vamos a usar en este capítulo, pero el dispositivo GPIO tiene
más puertos.

\begin{figure}[h]
  \centering
    \includegraphics[width=14cm]{graphs/gpio2.png}
  \caption{Otros puertos del GPIO (1ª parte)}
  \label{fig:gpio2}
\end{figure}

En la figura \ref{fig:gpio2} tenemos los siguientes:

\begin{itemize}
  \item {\tt GPLEVn}. Estos puertos devuelven el valor del pin respectivo. Si dicho pin está
        en torno a 0V devolverá un cero, si está en torno a 3.3V devolverá un 1.
  \item {\tt GPEDSn}. Sirven para detectar qué pin ha provocado una interrupción en caso de
        usarlo como lectura. Al escribir en ellos también podemos notificar que ya hemos procesado
        la interrupción y que por tanto estamos listos para que nos vuelvan a interrumpir sobre
        los pines que indiquemos.
  \item {\tt GPRENn}. Con estos puertos enmascaramos los pines que queremos que provoquen una
        interrupción en flanco de subida, esto es cuando hay una transición de 0 a 1 en el pin
        de entrada.
  \item {\tt GPFENn}. Lo mismo que el anterior pero en flanco de bajada.
\end{itemize}

El resto de puertos GPIO se muestran en la figura \ref{fig:gpio3}.

\begin{figure}[h]
  \centering
    \includegraphics[width=14cm]{graphs/gpio3.png}
  \caption{Otros puertos del GPIO (2ª parte)}
  \label{fig:gpio3}
\end{figure}

Estos registros son los siguientes:

\begin{itemize}
  \item {\tt GPHENn}. Enmascaramos los pines que provocarán una interrupción al detectar un
        nivel alto (3.3V) por dicho pin.
  \item {\tt GPLENn}. Lo mismo que el anterior pero para un nivel bajo (0V).
  \item {\tt GPARENn y GPAFENn}. Tienen funciones idénticas a {\tt GPRENn y GPFENn}, pero permiten
        detectar flancos en pulsos de poca duración.
  \item {\tt GPPUD y GPPUDCLKn}. Conectan resistencias de pull-up y de pull-down sobre los pines
        que deseemos. Para más información ver el último ejemplo del siguiente capítulo.
\end{itemize}

\subsection{Temporizador del sistema}

El temporizador del sistema es un reloj que funciona a 1MHz y en cada paso incrementa un
contador de 64bits. Este contador viene muy bien para implementar retardos o esperas
porque cada paso del contador se corresponde con un microsegundo. Los puertos asociados
al temporizador son los de la figura \ref{fig:systim}. Básicamente encontramos un contador
de 64 bits y cuatro comparadores. El contador está dividido en
dos partes, la parte baja {\tt CLO} y la parte alta {\tt CHI}. La parte alta no nos resulta
interesante, porque tarda poco más de una hora ($2^{32}$ \micro s) en incrementarse y no
va asociado a ningún comparador.

\begin{figure}[h]
  \centering
    \includegraphics[width=14cm]{graphs/systemtimer.png}
  \caption{System Timer}
  \label{fig:systim}
\end{figure}

Los comparadores son puertos que se pueden modificar y se comparan con {\tt CLO}. En el momento
que uno de los 4 comparadores coincida y estén habilitadas las interrupciones para dicho
comparador, se produce una interrupción y se activa el correspondiente bit {\tt Mx}
asociado al puerto {\tt CS} (para que en la rutina de tratamiento de interrupción o RTI
sepamos qué comparador ha provocado la interrupción).
Los comparadores {\tt C0} y {\tt C2} los emplea la GPU internamente, por lo que nosotros nos
ceñiremos a los comparadores {\tt C1} y {\tt C3}.

Las interrupciones las veremos en la siguiente lección. Por ahora sólo vamos
a acceder al puerto {\tt CLO} para hacer parpadear un LED a una frecuencia determinada. El
esquema funcional del {\tt System Timer} se muestra en la figura \ref{fig:esqtim}.

\begin{figure}[h]
  \centering
    \includegraphics[width=10cm]{graphs/esquematimer.png}
  \caption{Esquema funcional del System Timer}
  \label{fig:esqtim}
\end{figure}


\section{Ejemplos de programas Bare Metal}

\subsection{LED parpadeante con bucle de retardo}

La teoría sobre encender y apagar el LED la sabemos. Lo más sencillo que podemos hacer ahora
es hacer que el LED parpadee continuamente. Vamos a intruducir el siguiente programa en la
Raspberry, antes de probarlo piensa un poco cómo se comportaría el siguiente código.

\begin{lstlisting}[caption={esbn3.s},label={lst:codigoPract4_3}]
        .set    GPBASE,   0x20200000
        .set    GPFSEL0,  0x00
        .set    GPSET0,   0x1c
        .set    GPCLR0,   0x28
.text
        ldr     r0, =GPBASE
/* guia bits           xx999888777666555444333222111000*/
        mov     r1, #0b00001000000000000000000000000000
        str     r1, [r0, #GPFSEL0]  @ Configura como salida
/* guia bits           10987654321098765432109876543210*/
bucle:  mov     r1, #0b00000000000000000000001000000000
        str     r1, [r0, #GPSET0]   @ Enciende
        mov     r1, #0b00000000000000000000001000000000
        str     r1, [r0, #GPCLR0]   @ Apaga
        b       bucle
\end{lstlisting}

Para compilar y ejecutar este ejemplo sigue los pasos descritos en \ref{sec:Ejecutar}.
Al ejecutar el kernel.img resultante comprobamos que el LED no parpadea sino que está encendido con menos brillo del normal.
En realidad sí que lo hace, sólo que nuestro ojo es demasiado lento como para percibirlo.
Lo siguiente será ajustar la cadencia del parpadeo a un segundo para que podamos observar
el parpadeo. La secuencia sería apagar el LED, esperar medio segundo, encender el LED,
esperar otro medio segundo y repetir el bucle. Sabemos que el procesador de la Raspberry
corre a 700MHz por lo que vamos a suponer que tarde un ciclo de este reloj en ejecutar
cada instrucción. En base a esto vamos a crear dos bucles de retardo: uno tras apagar el LED
y otro tras encenderlo de 500ms cada uno. Un bucle de retardo lo único que hace es esperar
tiempo sin hacer realmente nada.

Si suponemos que cada instrucción consume un ciclo y teniendo en cuenta que el bucle de
retardo tiene 2 instrucciones, cada iteración del bucle consume 2 ciclos. A 700 MHz
($7\times10^{8}$ ciclos/segundo) un ciclo consume $1/({7\times10^{8})}$ segundos que es igual a
$1.42\times10^{-9} s$ (aproximadamente 1,5 ns). Así que cada iteración en principio
consume 3 ns y para consumir 500 ns necesitamos ${500\times10^{-3}/({3\times10^{-9}})}=166.66\times10^{6}$,
es decir más de 166 millones de iteraciones.

Si usamos ese número de iteraciones observaremos como la cadencia del
LED es más lenta de lo esperado, lo que quiere decir que cada iteración del bucle
de retardo tarda más de los dos ciclos que hemos supuesto.
Probamos con cronómetro en mano distintos valores para las constantes hasta comprobar que con
7 millones de iteraciones del bucle se consigue más o menos el medio segundo buscado. Haciendo
cuentas nos salen 50 ciclos por iteracción, bastante más de los 2 ciclos esperados. Esto se
debe a una dependencia de datos (ya que el flag que altera
la orden {\tt subs} es requerido justo después por la instrucción {\tt bne}) y que los saltos
condicionales suelen ser lentos.


\begin{lstlisting}[caption={Parte de esbn4.s},label={lst:codigoPract4_4}]
        .set    GPBASE,   0x20200000
        .set    GPFSEL0,  0x00
        .set    GPSET0,   0x1c
        .set    GPCLR0,   0x28
.text
        ldr     r0, =GPBASE
/* guia bits           xx999888777666555444333222111000*/
        mov     r1, #0b00001000000000000000000000000000
        str     r1, [r0, #GPFSEL0]  @ Configura GPIO 9
/* guia bits           10987654321098765432109876543210*/
        mov     r1, #0b00000000000000000000001000000000

bucle:  ldr     r2, =7000000
ret1:   subs    r2, #1              @ Bucle de retardo 1
        bne     ret1

        str     r1, [r0, #GPSET0]   @ Enciende el LED

        ldr     r2, =7000000
ret2:   subs    r2, #1              @ Bucle de retardo 2
        bne     ret2

        str     r1, [r0, #GPCLR0]   @ Enciende el LED

        b       bucle               @ Repetir para siempre
\end{lstlisting}

\subsection{LED parpadeante con temporizador}

Viendo lo poco preciso que es el temporizar con el bucle de retardo, vamos a sincronizar leyendo
continuamente el valor del {\tt System Timer}. Como el temporizador va a 1MHz, para temporizar
medio segundo lo único que tenemos que hacer es esperar a que el contador se incremente en
medio millón. El código final quedaría así.

\begin{lstlisting}[caption={esbn5.s},label={lst:codigoPract4_5}]
        .set    GPBASE,   0x20200000
        .set    GPFSEL0,        0x00
        .set    GPSET0,         0x1c
        .set    GPCLR0,         0x28
        .set    STBASE,   0x20003000
        .set    STCLO,          0x04
.text
        ldr     r0, =GPBASE
/* guia bits           xx999888777666555444333222111000*/
        mov     r1, #0b00001000000000000000000000000000
        str     r1, [r0, #GPFSEL0]  @ Configura GPIO 9
/* guia bits           10987654321098765432109876543210*/
        mov     r1, #0b00000000000000000000001000000000
        ldr     r2, =STBASE

bucle:  bl      espera        @ Salta a rutina de espera
        str     r1, [r0, #GPSET0]
        bl      espera        @ Salta a rutina de espera
        str     r1, [r0, #GPCLR0]
        b       bucle

/* rutina que espera medio segundo */
espera: ldr     r3, [r2, #STCLO]  @ Lee contador en r3
        ldr     r4, =500000
        add     r4, r3            @ r4= r3+medio millón
ret1:   ldr     r3, [r2, #STCLO]
        cmp     r3, r4            @ Leemos CLO hasta alcanzar
        bne     ret1              @ el valor de r4
        bx      lr
\end{lstlisting}

\subsection{Sonido con temporizador}

Este ejemplo es exactamente el mismo que el anterior, tan sólo hemos
cambiado el pin del LED (GPIO 9) por el pin asociado al altavoz de
nuestra placa de expansión (GPIO 4). También modificamos el tiempo
de espera para producir un sonido audible.

Vamos a producir un tono de 440 Hz. Para ello generamos una onda
cuadrada por dicho pin, que no es más que una serie de ceros y unos
consecutivos de idéntica duración. A esta duración la llamamos
semi-periodo, y es la que queremos calcular. Como el periodo es
el inverso de la frecuencia, tenemos que $periodo = 1/(440 s^{-1}) = 2.272\times10^{-3} s$,
por lo que el semi-periodo buscado es $2.272\times10^{-3}s/2=1.136\times10^{-3} s$ o lo
que es lo mismo, 1136 microsegundos.

\begin{lstlisting}[caption={Parte de esbn6.s},label={lst:codigoPract4_6}]
        ldr     r0, =GPBASE
/* guia bits           xx999888777666555444333222111000*/
        mov     r1, #0b00000000000000000001000000000000
        str     r1, [r0, #GPFSEL0]  @ Configura GPIO 4
/* guia bits           10987654321098765432109876543210*/
        mov     r1, #0b00000000000000000000000000010000
        ldr     r2, =STBASE

bucle:  bl      espera        @ Salta a rutina de espera
        str     r1, [r0, #GPSET0]
        bl      espera        @ Salta a rutina de espera
        str     r1, [r0, #GPCLR0]
        b       bucle

/* rutina que espera 1136 microsegundos */
espera: ldr     r3, [r2, #STCLO]  @ Lee contador en r3
        ldr     r4, =1136
        add     r4, r3            @ r4= r3 + 1136
ret1:   ldr     r3, [r2, #STCLO]
        cmp     r3, r4            @ Leemos CLO hasta alcanzar
        bne     ret1              @ el valor de r4
        bx      lr
\end{lstlisting}

\section{Ejercicios}

\subsection{Cadencia variable con bucle de retardo}

Usando la técnica del bucle de retardo haz que el LED parpadee
cada vez más rápido, hasta que la cadencia sea de 1/4 de segundo.
Una vez llegues a esta cadencia salta de golpe a la cadencia
original de 1 segundo. El tiempo que se tarda en pasar de una
cadencia a otra puede ser el que quieras, siempre que sea
suficiente para poder apreciar el efecto.

\subsection{Cadencia variable con temporizador}

Repite el ejercicio anterior pero empleando el temporizador
interno. Durante los 10 primeros segundos aumentamos la cadencia
del LED desde 1 segundo hasta los 250ms, y en los últimos 10
segundos disminuimos la cadencia al mismo ritmo de tal forma
que el ciclo completo se repite cada 20 segundos.

\subsection{Escala musical}

Escribe un programa que haga sonar el altavoz con las notas Do, Mi
y Sol (de la quinta octava) durante tres segundos cada una de ellas.
Las frecuencias de estas notas son:

\begin{longtable}{| p{2cm} | p{3cm} |}
\hline
{\bf Nota} & {\bf Frecuencia} \\ \hline
Do  & 523 Hz \\ \hline
Mi  & 659 Hz\\ \hline
Sol & 784 Hz\\ \hline
\end{longtable}

\chapterend{}
