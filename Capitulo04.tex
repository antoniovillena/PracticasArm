\documentclass[12pt,a4paper]{book} % article, report, book.
\input{A1.Preambulo.paquetes.tex}
\input{A2.Preambulo.commandos.tex}
\input{A3.Preambulo.entornos.tex}
\input{A4.Preambulo.estilodepagina.tex}
\usepackage{epstopdf}
\begin{document}

\chapterbegin{E/S a bajo nivel}
\label{chp:Subrut}
\minitoc

{\bf Objetivos}: Hasta ahora hemos programado en ensamblador sobre
la capa que nos ofrece el sistema operativo. Nosotros llamamos a una
función y ésta hace todo lo demás: le dice al sistema operativo lo
que tiene que hacer con tal periférico y el sistema operativo (en concreto
el kernel) le envía las órdenes directamente al periférico, al espacio
de memoria donde esté mapeado el mismo.

Lo que vamos a hacer en este capítulo es comunicarnos directamente con los
periféricos, para lo cual debemos prescindir totalmente del sistema operativo.
Este modo de acceder directamente al hardware de la máquina se denomina {\tt Bare Metal},
que traducido viene a ser algo como {\tt Metal desnudo}, haciendo referencia a
que estamos ante la máquina tal y cómo es, sin ninguna capa de abstracción de
por medio.

Veremos ejemplos de acceso directo a periféricos, en concreto el LED y los temporizadores,
que son bastante sencillos de manejar.


\section{Lectura previa}

\subsection{Librerías y Kernel, las dos capas que queremos saltarnos}

Anteriormente hemos utilizado funciones específicas para
comunicarnos con los periféricos. Si por ejemplo necesitamos escribir
en pantalla, llamamos a la función {\tt printf}. Pues bien, entre
la llamada a la función y lo que vemos en pantalla hay 2 capas de por medio.

Una primera capa se encuentra en la librería runtime que acompaña al
ejecutable, la cual incluye sólamente el fragmento de código de la
función que necesitemos, en este caso en {\tt printf}. El resto de
funciones de la librería ({\tt stdio}), si no las invocamos no aparecen
en el ejecutable. El enlazador se encarga de todo esto, tanto de ubicar
las funciones que llamemos desde ensamblador, como de poner la dirección
numérica correcta que corresponda en la instrucción {\tt bl printf}.

Este fragmento de código perteneciente a la primera capa sí que podemos
depurarlo mediante {\tt gdb}. Lo que hace es, a parte del formateo que
realiza la propia función, trasladar al sistema operativo una determinada
cadena para que éste lo muestre por pantalla. Es una especie de traductor
intermedio que nos facilita las cosas. Nosotros desde ensamblador también
podemos hacer llamadas al sistema directamente como veremos posteriormente.

La segunda capa va desde que hacemos la llamada al sistema hasta que se
produce la transferencia de datos al periférico, retornando desde la llamada
al sistema y volviendo a la primera capa, que a su vez retornará el control
a la llamada a librería que hicimos en nuestro programa inicialmente.

En esta segunda capa se ejecuta código del kernel, el cual no podemos depurar.
Además el procesador entra en un modo privilegiado, ya que en modo usuario (el
que se ejecuta en nuestro programa ensamblador y dentro de la librería) no
tenemos privilegios suficientes como para acceder a la zona de memoria que
mapea los periféricos.

Ahora veremos un ejemplo en el cual nos saltamos la capa intermedia para
comunicarnos directamente con el kernel vía llamada al sistema. En este ejemplo
vamos a escribir una simple cadena por pantalla, en concreto "Hola Mundo!".

\begin{lstlisting}[caption={esbn1.c},label={lst:codigoPract4_1},escapeinside={@}{@}]
.data

cadena: .asciz "Hola Mundo!\n"
cadenafin:

.text
.global main
 
main:   push    {r7, lr}           /* preservamos reg.*/
        mov     r0, #1             /* salida estándar */
        ldr     r1, =cadena        /* cadena a enviar */
        mov     r2, #cadenafin-cadena     /* longitud */
        mov     r7, #4             /* seleccionamos la*/
        swi     #0        /* llamada a sistema 'write'*/
        mov     r0, #0             /* devolvemos ok   */
        pop     {r7, lr}           /* recuperamos reg.*/
        bx      lr                 /* salimos de main */
\end{lstlisting}

La instrucción que ejecuta la llamada al sistema es {\tt swi \#0},
siempre tendrá cero como valor inmediato. El código numérico de
la llamada y el número de parámetros podemos buscarlo en cualquier
manual de Linux, buscando "Linux system call table" en Google. En
nuestro caso la llamada {\tt write} se corresponde con el código
4 y acepta tres parámetros: manejador de fichero, dirección de
los datos a escribir (nuestra cadena) y longitud de los datos.

En general se tiende a usar una lista reducida de posibles llamadas
a sistema, y que éstas sean lo más polivalentes posibles. En este
caso vemos que no existe una función específica para escribir en
pantalla. Lo que hacemos es escribir bytes en un fichero, pero usando
un manejador especial conocido como salida estándar, con lo cual todo
lo que escribamos a este fichero especial aparecerá por pantalla.

Pero el propósito de este capítulo no es saltarnos una capa
para comunicarnos directamente con el sistema operativo. Lo que queremos
es saltarnos las dos capas y enviarle órdenes directamente a los periféricos.
Para esto tenemos que saltarnos el sistema operativo, o lo que es lo mismo,
hacer nosotros de sistema operativo para realizar las tareas que queramos.

Este modo de trabajar (como hemos adelantado) se denomina Bare Metal, porque
accedemos a las entrañas del hardware. En él podemos hacer desde cosas
muy sencillas como encender un LED hasta programar desde cero nuestro propio
sistema operativo.

\subsection{Ejecutar código en Bare Metal}

El proceso de arranque de la Raspberry Pi es el siguiente:

\begin{itemize}
  \item Cuando la encendemos, el núcleo ARM está desactivado. Lo primero que se activa es el
        núcleo GPU, que es un procesador totalmente distinto e independiente al ARM. En este
        momento la SDRAM está desactivada.
  \item El procesador GPU empieza a ejecutar la primera etapa del bootloader (son 3 etapas), que
        está almacenada en ROM dentro del mismo chip que comparten ARM y GPU. Esta primera etapa
        accede a la tarjeta SD y lee el fichero {\tt bootcode.bin} en caché L2 y lo ejecuta,
        siendo el código de {\tt bootcode.bin} la segunda etapa del bootloader.
  \item En la segunda etapa se activa la SDRAM y se carga la tercera parte del bootloader, cuyo
        código está repartido entre {\tt loader.bin} (opcional) y {\tt start.elf}.
  \item En tercera y última etapa del bootloader se accede opcionalmente a dos archivos ASCII de
        configuración llamados {\tt config.txt} y {\tt cmdline.txt}. Lo más relevante de esta
        etapa es que cargamos en RAM (en concreto en la dirección 0x8000) el archivo
        {\tt kernel.img} con código ARM, para luego ejecutarlo y acabar con el bootloader, pasando
        el control desde la GPU hacia la CPU.
        Este último archivo es el que nos interesa modificar para nuestros propósitos, ya que es
        lo primero que la CPU ejecuta y lo hace en modo privilegiado, es decir, con acceso total
        al hardware.
\end{itemize}

De todos estos archivos los obligatorios son {\tt bootcode.bin}, {\tt start.elf} y
{\tt kernel.img}. Los dos primeros los bajamos del repositorio oficial
\footnote{\url{https://github.com/raspberrypi}} y el tercero {\tt kernel.img} es el que nosotros
vamos a generar. Estos tres archivos deben estar en el directorio raíz de la primera partición
de la tarjeta SD, la cual debe estar formateada en FAT32.

El proceso completo que debemos repetir cada vez que desarrollemos un programa nuevo
en Bare Metal es el siguiente:

\begin{itemize}
  \item Apagamos la Raspberry.
  \item Extraemos la tarjeta SD.
  \item Introducimos la SD en el lector de nuestro ordenador de desarrollo.
  \item Montamos la unidad y copiamos (sobreescribimos) el kernel.img que acabamos
        de desarrollar.
  \item Desmontamos y extraemos la SD.
  \item Insertamos de nuevo la SD en la Raspberry y la encendemos.
\end{itemize}

Es un proceso sencillo para las prácticas que vamos a hacer, pero para proyectos más largos
se vuelve bastante pesado de realizar. Hay varias alternativas que agilizan el ciclo de
trabajo, donde no es necesario extraer la SD y por tanto podemos actualizar el {\tt kernel.img}
en cuestión de segundos.

\begin{itemize}
  \item Cable JTAG con software Openocd \footnote{\url{http://openocd.sourceforge.net}}
  \item Cable USB-serie desde el ordenador de desarrollo hacia la Raspberry, requiere
        tener instaladas las herramientas de compilación cruzada en el ordenador de desarrollo.
  \item Cable serie-serie que comunica dos Raspberries, una orientada a desarrollo y la otra
        para ejecutar los programas en Bare Metal. No es imprescindible trabajar directamente
        con la Raspberry de desarrollo, podemos acceder vía ssh con nuestro ordenador habitual,
        sin necesidad de tener instaladas las herramientas de compilación en el mismo.
\end{itemize}

En las dos últimas opciones lo que se almacena en el kernel.img de la SD es un bootloader
que lee continuamente del puerto serie y en el momento en que recibe un archivo lo carga
en RAM y lo ejecuta. Para más información sobre estos métodos recomendamos que sigan los
README del siguiente repositorio \footnote{\url{https://github.com/dwelch67/raspberrypi}}.

\subsection{Acceso a periféricos}

Los periféricos se controlan leyendo y escribiendo datos a los registros asociados. No
confundir estos registros con los registros de la CPU. Un registro asociado a un periférico
es un ente, normalmente del mismo tamaño que el ancho del bus de datos, que sirve para
configurar diferentes aspectos del mismo. No se trata de RAM, por lo que no se garantiza que
al leer de un registro obtengamos el último valor que escribimos. Es más, incluso hay
registros que sólo admiten ser leídos y otros que sólo admiten escrituras. La funcionalidad
de los registros también es muy variable, incluso dentro de un mismo registro los diferentes
bits del mismo tienen distinto comportamiento.

Como cada periférico se controla de una forma diferente, no hay más remedio que leerse
el datasheet del mismo si queremos trabajar con él. En nuestro caso queremos encender el LED
de la Raspberry que está conectado al dispositivo GPIO (entrada/salida de propósito general).
Este dispositivo tiene puertos accesibles al exterior mediante un conector de dos filas de
pines en la esquina superior izquierda de la Raspberry, por lo que podemos conseguir los
mismos resultados conectando LEDs sobre diferentes pines de este conector.

El datasheet que tenemos que buscar es el siguiente
\footnote{\url{http://www.raspberrypi.org/wp-content/uploads/2012/02/BCM2835-ARM-Peripherals.pdf}},
ya que el dispositivo GPIO se encuentra en el propio chip (el mismo que contiene CPU y GPU).
En el primer capítulo nos encontramos con el siguiente párrafo.

{\it Physical addresses range from 0×20000000 to 0x20FFFFFF for peripherals. The bus addresses
for peripherals are set up to map onto the peripheral bus address range starting at
0x7E000000. Thus a peripheral advertised here at bus address 0x7Ennnnnn is available
at physical address 0x20nnnnnn.}

Esto quiere decir que los registros de los periféricos están mapeados en memoria en
0×20nnnnnn, que es donde haremos las lecturas y escrituras. Pero debemos
trasladar las direcciones que nos indica el datasheet desde 0x7Ennnnnn hasta 0x20nnnnnn,
porque lo primero son direcciones del bus y lo segundo direcciones físicas, y desde
software empleamos estas últimas.

Para nuestros propósitos nos bastaría con acceder a los registros GPFSELx, GPSETx y GPCLRx,
que se corresponden con 13 de los 41 registros que contiene el GPIO. Vemos la tabla con las
direcciones de estos registros (dos de los cuales no se usan, están reservados)

\begin{longtable}{ p{1.8cm} | p{2cm} | p{5cm} | p{1cm} | p{1cm} }
\hline
{\bf Dirección} & {\bf Nombre} & {\bf Descripción} & {\bf Tam.} & {\bf Tipo} \\ \hline
7E200000 & GPFSEL0 & GPIO Function Select 0 & 32 & R/W \\ \hline
7E200004 & GPFSEL1 & GPIO Function Select 1 & 32 & R/W \\ \hline
7E200008 & GPFSEL2 & GPIO Function Select 2 & 32 & R/W \\ \hline
7E20000C & GPFSEL3 & GPIO Function Select 3 & 32 & R/W \\ \hline
7E200010 & GPFSEL4 & GPIO Function Select 4 & 32 & R/W \\ \hline
7E200014 & GPFSEL5 & GPIO Function Select 5 & 32 & R/W \\ \hline
7E200018 & -       & Reservado              & -  & -   \\ \hline
7E20001C & GPSET0  & GPIO Pin Output Set 0  & 32 & W   \\ \hline
7E200020 & GPSET1  & GPIO Pin Output Set 1  & 32 & W   \\ \hline
7E200024 & -       & Reservado              & -  & -   \\ \hline
7E200028 & GPCLR0  & GPIO Pin Output Clear 0 & 32 & W  \\ \hline
7E20002C & GPCLR1  & GPIO Pin Output Clear 1 & 32 & W  \\ \hline
\end{longtable}

Estos registros controlan 54 pines, que a su vez están agrupados
en 6 grupos funcionales de 9 pines cada uno. El LED interno de la
Raspberry se corresponde con el pin 6 del GPIO, se nombran con
GPIO más el número de pin, en nuestro caso sería GPIO 6. Nótese
que la numeración empieza en 0, desde GPIO 0 hasta GPIO 53. De
todos estos pines sólo diecisiete (0, 1, 4, 7, 8, 9, 10, 11, 14, 15, 17, 18, 22, 23, 24, 25 y 27)
son accesibles a través del conector, y si disponemos del modelo B+ la lista
aumenta en nueve (5, 6, 12, 13, 16, 19, 20, 21 y 26).






\chapterend{}

%%%%%%%%%%%%%%%%%%%%%%%%%%%%%%%%%%%%%%%%%%%%%%%%%%%%%%%%%%%%%%
\end{document}