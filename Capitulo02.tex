\documentclass[12pt,a4paper]{book} % article, report, book.
\input{A1.Preambulo.paquetes.tex}
\input{A2.Preambulo.commandos.tex}
\input{A3.Preambulo.entornos.tex}
\input{A4.Preambulo.estilodepagina.tex}
\usepackage{epstopdf}
\begin{document}

%%%%%%%%%%%%%%%%%%%%%%%%%%%%%%%%%%%%%%%%%%%%%%%%%%%%%%%%%%%%%%

\chapterbegin{Tipos de datos y sentencias de alto nivel}
\label{chp:TipDat}
\minitoc

{\bf Objetivo}:
En esta sesión repasaremos cómo se representa la
información en la memoria del computador: veremos la definición en
ensamblador de punteros, vectores y matrices. También veremos cómo se programan las
estructuras de alto nivel del tipo {\tt if-else} y los bucles {\tt
for} y {\tt while}.

\section{Lectura previa}

\subsection{Modos de direccionamiento del ARM}

En la arquitectura ARM los accesos a memoria se hacen mediante instrucciones
específicas {\tt ldr} y {\tt str} (luego veremos las variantes {\tt ldm},
{\tt stm} y las preprocesadas {\tt push} y {\tt pop}). El resto de instrucciones
toman operandos desde registros o valores inmediatos, sin excepciones. En este
caso la arquitectura nos fuerza a que trabajemos de un modo determinado: primero
cargamos los registros desde memoria, luego procesamos el valor de estos registros
con el amplio abanico de instrucciones del ARM, para finalmente volcar los
resultados desde registros a memoria. Existen otras arquitecturas como la Intel x86,
donde las instrucciones de procesado nos permiten leer o escribir directamente
de memoria. Ningún método es mejor que otro, todo es cuestión de diseño. Normalmente
se opta por direccionamiento a memoria en instrucciones de procesado en arquitecturas
con un número reducido de registros, donde se emplea la memoria como almacén
temporal. En nuestro caso disponemos de suficientes registros, por lo que podemos
hacer el procesamiento sin necesidad de interactuar con la memoria, lo que por otro
lado también es más rápido.

\begin{descript}
  \item[Direccionamiento inmediato.]
    El operando fuente es una constante, formando parte de la instrucción.
\begin{lstlisting}
    mov     r0, #1
    add     r2, r3, #4
\end{lstlisting}
  \item[Direccionamiento inmediato con desplazamiento o rotación.]
    Es una variante del anterior en la cual se permiten operaciones intermedias
    sobre los registros.
\begin{lstlisting}
    mov     r1, r2, LSL #1      /* r1 <- (r2*2) */
    mov     r1, r2, LSL #2      /* r1 <- (r2*4) */
    mov     r1, r3, ASR #3      /* r1 <- (r3/8) */
\end{lstlisting}
    Implicitamente también interviene en la creación de constantes, rotando o
    desplazando constantes más pequeñas de forma transparente al usuario. Como todas
    las instrucciones ocupan 32 bits, es técnicamente imposible que podamos cargar
    en un registro cualquier constante de 32 bits con la instrucción {\tt mov}. Por
    esta razón cuando se necesita cargar una constante más compleja en un registro
    (como una dirección a una variable de memoria) se emplea {\tt ldr}. En algunos
    casos es difícil determinar si el ensamblador conseguirá codificar una constante
    por medio de esta técnica, con lo que la única solución es probar.
\begin{lstlisting}
    mov     r1, #0x80000020
\end{lstlisting}
    Ensamblamos y vemos que no da problemas. Sin embargo con esta otra.
\begin{lstlisting}
    mov     r1, #0x80000040
\end{lstlisting}
    El ensamblador nos muestra el siguiente error.
\begin{lstlisting}
intro1.s: Assembler messages:
intro1.s:10: Error: invalid constant (80000040) after fixup
\end{lstlisting}

  \item[Direccionamiento a memoria, sin actualizar registro puntero.]
    Es la forma más sencilla y admite 4 variantes. Después del acceso
    a memoria ningún registro implicado en el cálculo de la dirección
    se modifica.
\begin{itemize}
  \item{\tt [Rx, +\#inmediato] \newline
            [Rx, -\#inmediato] \newline}
    Simplemente añade (o sustrae) un valor inmediato al registro dado
    para calcular la dirección. Es muy útil para acceder a elementos
    fijos de un array, ya que el desplazamiento es constante. Por
    ejemplo si tenemos {\tt r1} apuntando a un array de enteros de
    32 bits {\tt int a[]} y queremos poner a 1 el
    elemento {\tt a[3]}, lo hacemos así.

\begin{lstlisting}
    mov     r2, #1          /* r2 <- 1          */
    str     r2, [r1, #+12]  /* *(r1 + 12) <- r2 */
\end{lstlisting}

    Nótese que hemos multiplicado por 4 el desplazamiento porque cada
    elemento del array son 4 bytes. El desplazamiento no puede ser mayor
    de 12 bits, por lo que nuestro rango está límitado entre {\tt [Rx, \#-4095]}
    y {\tt [Rx, \#+4095]}.

  \item{\tt [Rx, +Ry] \newline
            [Rx, -Ry] \newline}

    Parecido al anterior pero en lugar de un inmediato emplea otro registro. Útil
    en el caso de queramos mantener fijo el registro {\tt Rx} y movernos con {\tt Ry},
    o bien para acceder a desplazamientos mayores a 4095. El mismo ejemplo de
    arriba utilizando esta variante sería.

\begin{lstlisting}
    mov     r2, #1          /* r2 <- 1          */
    mov     r3, #12         /* r3 <- 12         */
    str     r2, [r1, +r3]   /* *(r1 + r3) <- r2 */
\end{lstlisting}

  \item{\tt [Rx, +Ry, operación\_desp \#inmediato] \newline
            [Rx, -Ry, operación\_desp \#inmediato] \newline}

    En este caso aplicamos una operación de desplazamiento
    o rotación sobre el segundo registro {\tt Ry}. Muy útil
    en caso de arrays o estructuras con elementos de longitud
    potencia de 2, ya que podemos indexar directamente. El
    mismo ejemplo de antes.
    
\begin{lstlisting}
    mov     r2, #1
    mov     r3, #3
    str     r2, [r1, +r3, LSL #2]
\end{lstlisting}

    Nótese cómo accedemos a {\tt a[3]}
    directamente con el valor del índice, {\tt 3}).
\end{itemize}


  \item[Direccionamiento a memoria, actualizando registro puntero.]
    Es la forma más sencilla y admite 4 variantes. Después del acceso
    a memoria ningún registro implicado en el cálculo de la dirección
    se modifica.


\end{descript}


\chapterend{}

%%%%%%%%%%%%%%%%%%%%%%%%%%%%%%%%%%%%%%%%%%%%%%%%%%%%%%%%%%%%%%
\end{document}