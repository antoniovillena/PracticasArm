\documentclass[12pt,a4paper]{book} % article, report, book.
\input{A1.Preambulo.paquetes.tex}
\input{A2.Preambulo.commandos.tex}
\input{A3.Preambulo.entornos.tex}
\input{A4.Preambulo.estilodepagina.tex}
\usepackage{epstopdf}
\begin{document}

%%%%%%%%%%%%%%%%%%%%%%%%%%%%%%%%%%%%%%%%%%%%%%%%%%%%%%%%%%%%%%

\chapterbegin{Tipos de datos y sentencias de alto nivel}
\label{chp:TipDat}
\minitoc

{\bf Objetivo}:
En esta sesión repasaremos cómo se representa la
información en la memoria del computador: veremos la definición en
ensamblador de punteros, vectores y matrices. También veremos cómo se programan las
estructuras de alto nivel del tipo {\tt if-else} y los bucles {\tt
for} y {\tt while}.

\section{Lectura previa}

\subsection{Modos de direccionamiento del ARM}

En la arquitectura ARM los accesos a memoria se hacen mediante instrucciones
específicas {\tt ldr} y {\tt str} (luego veremos las variantes {\tt ldm},
{\tt stm} y las preprocesadas {\tt push} y {\tt pop}). El resto de instrucciones
toman operandos desde registros o valores inmediatos, sin excepciones. En este
caso la arquitectura nos fuerza a que trabajemos de un modo determinado: primero
cargamos los registros desde memoria, luego procesamos el valor de estos registros
con el amplio abanico de instrucciones del ARM, para finalmente volcar los
resultados desde registros a memoria. Existen otras arquitecturas como la Intel x86,
donde las instrucciones de procesado nos permiten leer o escribir directamente
de memoria. Ningún método es mejor que otro, todo es cuestión de diseño. Normalmente
se opta por direccionamiento a memoria en instrucciones de procesado en arquitecturas
con un número reducido de registros, donde se emplea la memoria como almacén
temporal. En nuestro caso disponemos de suficientes registros, por lo que podemos
hacer el procesamiento sin necesidad de interactuar con la memoria, lo que por otro
lado también es más rápido.

\begin{descript}
  \item[Direccionamiento inmediato.]
    El operando fuente es una constante, formando parte de la instrucción.
\begin{lstlisting}
    mov     r0, #1
    add     r2, r3, #4
\end{lstlisting}
  \item[Direccionamiento inmediato con desplazamiento o rotación.]
    Es una variante del anterior en la cual se permiten operaciones intermedias
    sobre los registros.
\begin{lstlisting}
    mov     r1, r2, LSL #1      /* r1 <- (r2*2) */
    mov     r1, r2, LSL #2      /* r1 <- (r2*4) */
    mov     r1, r3, ASR #3      /* r1 <- (r3/8) */
\end{lstlisting}
    Implicitamente también interviene en la creación de constantes, rotando o
    desplazando constantes más pequeñas de forma transparente al usuario. Como todas
    las instrucciones ocupan 32 bits, es técnicamente imposible que podamos cargar
    en un registro cualquier constante de 32 bits con la instrucción {\tt mov}. Por
    esta razón cuando se necesita cargar una constante más compleja en un registro
    (como una dirección a una variable de memoria) se emplea {\tt ldr}. En algunos
    casos es difícil determinar si el ensamblador conseguirá codificar una constante
    por medio de esta técnica, con lo que la única solución es probar.
\begin{lstlisting}
    mov     r1, #0x80000020
\end{lstlisting}
    Ensamblamos y vemos que no da problemas. Sin embargo con esta otra.
\begin{lstlisting}
    mov     r1, #0x80000040
\end{lstlisting}
    El ensamblador nos muestra el siguiente error.
\begin{lstlisting}
intro1.s: Assembler messages:
intro1.s:10: Error: invalid constant (80000040) after fixup
\end{lstlisting}

  \item[Direccionamiento a memoria, sin actualizar registro puntero.]
    Es la forma más sencilla y admite 4 variantes. Después del acceso
    a memoria ningún registro implicado en el cálculo de la dirección
    se modifica.
\begin{itemize}
  \item{\tt [Rx, \#+inmediato] \newline
            [Rx, \#-inmediato] \newline}
    Simplemente añade (o sustrae) un valor inmediato al registro dado
    para calcular la dirección. Es muy útil para acceder a elementos
    fijos de un array, ya que el desplazamiento es constante. Por
    ejemplo si tenemos {\tt r1} apuntando a un array de enteros de
    32 bits {\tt int a[]} y queremos poner a 1 el
    elemento {\tt a[3]}, lo hacemos así.

\begin{lstlisting}
    mov     r2, #1          /* r2 <- 1          */
    str     r2, [r1, #+12]  /* *(r1 + 12) <- r2 */
\end{lstlisting}

    Nótese que hemos multiplicado por 4 el desplazamiento porque cada
    elemento del array son 4 bytes. El desplazamiento no puede ser mayor
    de 12 bits, por lo que nuestro rango está límitado entre {\tt [Rx, \#-4095]}
    y {\tt [Rx, \#+4095]}.

  \item{\tt [Rx, +Ry] \newline
            [Rx, -Ry] \newline}

    Parecido al anterior pero en lugar de un inmediato emplea otro registro. Útil
    en el caso de queramos mantener fijo el registro {\tt Rx} y movernos con {\tt Ry},
    o bien para acceder a desplazamientos mayores a 4095. El mismo ejemplo de
    arriba utilizando esta variante sería.

\begin{lstlisting}
    mov     r2, #1          /* r2 <- 1          */
    mov     r3, #12         /* r3 <- 12         */
    str     r2, [r1, +r3]   /* *(r1 + r3) <- r2 */
\end{lstlisting}

  \item{\tt [Rx, +Ry, operación\_desp \#inmediato] \newline
            [Rx, -Ry, operación\_desp \#inmediato] \newline}

    En este caso aplicamos una operación de desplazamiento
    o rotación sobre el segundo registro {\tt Ry}. Muy útil
    en caso de arrays o estructuras con elementos de longitud
    potencia de 2, ya que podemos indexar directamente. El
    mismo ejemplo de antes.
    
\begin{lstlisting}
    mov     r2, #1
    mov     r3, #3
    str     r2, [r1, +r3, LSL #2]
\end{lstlisting}

    Nótese cómo accedemos a {\tt a[3]}
    directamente con el valor del índice, {\tt 3}).
\end{itemize}


  \item[Direccionamiento a memoria, actualizando registro puntero.]
    En este modo de direccionamiento, el registro que genera la dirección
    se actualiza con la propia dirección. De esta forma podemos recorrer
    un array con un sólo registro sin necesidad de hacer el incremento del
    puntero en una instrucción aparte. Hay dos métodos de actualizar dicho
    registro, antes de ejecutar la instrucción (preindexado) o después
    de la misma (postindexado). Los tres siguientes tipos son los
    postindexados.

\begin{itemize}
  \item{\tt [Rx], \#+inmediato \newline
            [Rx], \#-inmediato \newline}
    Una notación muy parecida a la versión que no actualiza registro, la única
    diferencia es que la constante de desplazamiento queda fuera de los corchetes.
    Presenta el mismo límite de hasta 4095. Este ejemplo pone a cero los 3 primeros
    elementos {\tt a[0], a[1], a[2]} del array.

\begin{lstlisting}
    mov     r2, #0          /* r2 <- 0      */
    str     r2, [r1], #+4   /* a[0] <- r2   */
    str     r2, [r1], #+4   /* a[1] <- r2   */
    str     r2, [r1], #+4   /* a[2] <- r2   */
\end{lstlisting}

  \item{\tt [Rx], +Ry \newline
            [Rx], -Ry \newline}
    Igual que antes pero con registro en lugar de inmediato.

  \item{\tt [Rx], +Ry, operación\_desp \#inmediato \newline
            [Rx], -Ry, operación\_desp \#inmediato \newline}

    Nótese que en todos los modos postindexados
    encerramos entre llaves el primer registro, que es el que se va
    a utilizar en la instrucción de lectura o escritura en memoria. Es decir
    primero cargamos de {\tt [Rx]} y luego actualizamos {\tt Rx} con el valor
    que corresponda. Esta instrucción.

\begin{lstlisting}
    ldr     r2, [r1], +r3, LSL #2
\end{lstlisting}

    Se puede desglosar en estas otras dos, cuyo comportamiento es exactamente
    el mismo.

\begin{lstlisting}
    ldr     r2, [r1]
    add     r1, r1, r3, LSL #2
\end{lstlisting}
\end{itemize}

    Ya hemos visto la notación postindexada. Veamos ahora los tres modos
    preindexados.

\begin{itemize}
  \item{\tt [Rx, \#+inmediato]! \newline
            [Rx, \#-inmediato]! \newline}
    La idea en todos los casos es encerrar entre corchetes la dirección que
    se va a usar en la instrucción. Para diferenciarlo del caso que no actualiza
    el registro le añadimos un {\tt !} al final.

    Este modo es muy útil en casos que queramos reusar en una futura instrucción
    la dirección que hemos calculado. En este ejemplo duplicamos el valor
    que se encuentra en {\tt a[3]}.

\begin{lstlisting}
    ldr     r2, [r1, #+12]!
    add     r2, r2, r2
    str     r2, [r1]
\end{lstlisting}

  \item{\tt [Rx, +Ry]! \newline
            [Rx, -Ry]! \newline}

    Similar al anterior pero usando {\tt Ry} en lugar de inmediato.

  \item{\tt [Rx, +Ry, operación\_desp \#inmediato]! \newline
            [Rx, -Ry, operación\_desp \#inmediato]! \newline}

    Tercer y último caso de direccionamiento preindexado. Al igual que
    antes, desgloso en dos instrucciones para ver el funcionamiento exacto.

\begin{lstlisting}
    ldr     r2, [r1, +r3, LSL #2]!
\end{lstlisting}

    Equivale a esto.

\begin{lstlisting}
    ldr     r2, [r1, +r3, LSL #2]
    add     r1, r1, r3, LSL #2
\end{lstlisting}

    O bien a esto otro.

\begin{lstlisting}
    add     r1, r1, r3, LSL #2
    ldr     r2, [r1]
\end{lstlisting}
\end{itemize}


\subsection{Tipos de datos}

\vspace{0.25cm}
{\bf Tipos de datos básicos.} En la siguiente tabla se recogen los diferentes tipos de datos básicos
 que podrán aparecer en los ejemplos, así como su tamaño y  rango de
representación.

\vspace{1cm}
\begin{center}
%\begin{table}[h!]
\begin{tabular}{|l|l|c|c|}\hline
 ARM & Tipo en C & bits & Rango \\ \hline
 {\tt .byte}  & {\tt unsigned char} & 8 & 0 a 255 \\
              & ({\tt signed}) {\tt char} & 8 & -128 a 127 \\ \hline
 {\tt .hword} & {\tt unsigned short int} & 16 &  0 a 65.535 \\
 {\tt .short} & ({\tt signed}) {\tt short int} & 16 & -32.768 a 32767 \\ \hline
 {\tt .word}  & {\tt unsigned int} & 32 & 0 a 65.535 \\
 {\tt .int}   & ({\tt signed}) {\tt int} &  32 & -32.768 a 32.767 \\
              & {\tt unsigned long int}  & 32 & 0 a 4294967296 \\ 
              & ({\tt signed}) {\tt long int} &  32 & -2147483648 a 2147483647 \\ \hline
 {\tt .quad}  & {\tt unsigned long long} & 64 &  0 a $2^{64}$ \\
              & ({\tt signed}) {\tt long long} & 64 & -$2^{63}$ a $2^{63}$-1 \\ \hline
\end{tabular}
%\end{table}
\end{center}

Nótese como en ensamblador los tipos son neutrales al signo, lo importante
es la longitud en bits del tipo. La mayoría de las instrucciones (salvo
multiplicación) hacen la misma operación tanto si se trata de un número
natural como si es entero en complemento a dos. Nosotros decidiremos el tipo
mediante las constantes que pongamos o según los flags que interpretemos del
resultado de la operación.
\end{descript}


\noindent{\bf Punteros.} Un {\bf puntero} siempre ocupa 32 bits y contiene
una dirección de memoria. Los punteros siempre se cargan con una instrucción
{\tt ldr}, y su valor se obtiene normalmente con un direccionamiento relativo
a {\tt pc}. Estas direcciones se almacenan justo después de la instrucción
de retorno de nuestra función {\tt bx lr}, algo como esto.

\begin{lstlisting}
.data
var1:   .word   3

.text
.global main
main:   ldr     r1, puntero_var1    /* r1 <- &var1    */
        ldr     r2, [r1]            /* r2 <- *r1      */
        bx      lr
puntero_var1:   .word   var1
\end{lstlisting}

En el primer {\tt ldr} cargamos el valor del puntero en el registro {\tt r1}.
En el segundo {\tt ldr} leemos en {\tt r2} el valor de la variable apuntada por el puntero,
en este caso, {\tt 3}. No podemos escribir sobre {\tt puntero\_var1}, recuerda
que estamos en la sección de código, por razones de seguridad saltaría una excepción si
intentamos escribir ahí.

Normalmente se emplear la notación {\tt =variable} en el primer {\tt ldr}, por lo que el
ensamblador se encargaría de hacer el trabajo sucio de alojar la constante del puntero
debajo de la instrucción de retorno de la función.

\begin{lstlisting}
.data
var1:   .word   3

.text
.global main
main:   ldr     r1, =var1    /* r1 <- &var1    */
        ldr     r2, [r1]     /* r2 <- *r1      */
        bx      lr
\end{lstlisting}


\noindent{\bf Vectores.} Todos los elementos de un vector se almacenan en un único
bloque de memoria a partir de una dirección determinada. Los
diferentes elementos se almacenan en posiciones consecutivas, de
manera que el elemento {\tt i}  está entre los {\tt i-1} e {\tt i+1}
(figura \ref{fig:2.3}). Los
vectores están definidos siempre a partir de la posición 0. El propio
índice indica cuántos elementos hemos de desplazarnos respecto del
comienzo del primer elemento (para acceder al elemento cero hemos de
saltarnos 0 elementos, para acceder al elemento 1 hemos de saltarnos
un elemento, etc... En general, para acceder al elemento con índice
{\tt i}
hemos de saltarnos los {\tt i} elementos anteriores).

Dado un vector {\tt int v[N];}, todos los elementos se encuentran en posiciones
consecutivas a partir de la dirección de {\tt v[0]}
(puesto que son {\tt int}, en este ejemplo, cada elemento ocupa 4 bytes). Por lo tanto,
el acceso al elemento {\tt v[i]} se consigue aplicando la siguiente expresión.

\begin{equation}
v[i] = M_d[@v[0] + i*4] 
\label{eq:uno}
\end{equation}

Con $ @v[0]$ nos referimos a la dirección en memoria del elemento
 $ v[0]$. Con $ M_d[\ ]$ notamos el acceso a memoria para la lectura/escritura
de un dato (el número de bytes de memoria implicado dependerá del tipo
de datos declarado). Cuando nos queramos refererir al acceso a memoria
para la obtención de un puntero, lo notaremos como
$ M_{ref}[\ ]$.

\begin{figure}[h]
  \centering
    \includegraphics[width=10cm]{graphs/2-3.eps}
  \caption{Representación de un vector en memoria}
  \label{fig:2.3}
\end{figure}


\noindent{\bf Matrices bidimensionales.} Una matriz bidimensional de N$\times$M
elementos se almacena en un único bloque de memoria. Interpretaremos
una matriz de N$\times$M como una matriz con $N$ filas de $M$ elementos cada
una. Si cada elemento de la matriz ocupa $B$ bytes, la matriz ocupará un
bloque de  $M \times N \times B$ bytes (ver figura \ref{fig:2.5}(a)).

Dentro de este bloque, los elementos se almacenan por filas. Primero
se guardan todos los elementos de la fila $0$, después todos los de la
fila $1$, etc... Dentro de cada fila, los elementos están ordenados por
columnas (figura \ref{fig:2.5}(b)).


\begin{figure}[h]
  \centering
    \includegraphics[width=10cm]{graphs/2-5.eps}
  \caption{(a) Formato de una matriz  C con $N$ filas y $M$ columnas y (b)
organización por filas}
  \label{fig:2.3}
\end{figure}

Por lo tanto, para acceder al elemento {\tt mat[i][j]} hemos de saltar
$i$ filas completas (de $M$ elementos de $B$ bytes) y después $j$
elementos de $B$ bytes (suponiendo una matriz de enteros, $B= 2$ bytes). Es
decir, la fórmula para obtener el elemento {\tt mat[i][j]} será:
\begin{equation}
mat[i][j] = M_d[@mat + ((i*M)+j)*B] 
\label{eq:accesoelemmatriz}
\end{equation}


\subsection{Instrucciones de salto}

Las instrucciones de salto pueden producir saltos incondicionales ({\tt b} y {\tt bx})
o saltos condicionales. En los saltos condicionales añadimos dos o tres letras
a la ({\tt b/bx}), mediante las cuales condicionamos si se salta o no dependiendo
del estado de los flags. Estas condiciones se pueden añadir a cualquier
otra instrucción, aunque la mayoría de las veces lo que nos interesa es controlar
el flujo del programa y así ejecutar o no un grupo de instrucciones dependiendo
del resultado de una operación (reflejado en los flags).

La lista completa de condiciones es ésta.

\begin{itemize}
  \item{\tt EQ} ({\tt eq}ual, igual). Cuando {\tt Z} está activo ({\tt Z} vale 1).
  \item{\tt NEQ} ({\tt n}ot {\tt eq}ual, igual). Cuando {\tt Z} está inactivo ({\tt Z} vale 0).
  \item{\tt MI} ({\tt mi}nus, negativo). Cuando {\tt N} está activo ({\tt N} vale 1).
  \item{\tt PL} ({\tt pl}us, positivo o cero). Cuando {\tt N} está inactivo ({\tt N} vale 0).
  \item{\tt CS/HS} ({\tt c}arry {\tt s}et/{\tt h}igher or {\tt s}ame, carry activo/mayor o igual). Cuando {\tt C} está activo ({\tt C} vale 1).
  \item{\tt CC/LO} ({\tt c}arry {\tt c}lear/{\tt l}ower, carry inactivo/menor). Cuando {\tt C} está inactivo ({\tt C} vale 0).
  \item{\tt VS} (o{\tt v}erlow {\tt s}et, desbordamiento activo). Cuando {\tt V} está activo ({\tt V} vale 1).
  \item{\tt VC} (o{\tt v}erlow {\tt c}lear, desbordamiento inactivo). Cuando {\tt V} está inactivo ({\tt V} vale 0).
  \item{\tt GT} ({\tt g}reater {\tt t}han, mayor en complemento a dos). Cuando {\tt Z} está inactivo y {\tt N=V}({\tt Z} vale 0, {\tt N} vale {\tt V}).
  \item{\tt LT} ({\tt l}ower {\tt t}han, menor en complemento a dos). Cuando {\tt N!=V} ({\tt N} vale {\tt not V}).
  \item{\tt GE} ({\tt g}reater or {\tt e}qual, mayor o igual en complemento a dos). Cuando {\tt N=V}({\tt N} vale {\tt V}).
  \item{\tt LE} ({\tt l}ower or {\tt e}qual, menor o igual en complemento a dos). Cuando {\tt Z} está activo y {\tt N!=V} ({\tt Z} vale 1, {\tt N} vale {\tt not V}).
  \item{\tt HI} ({\tt h}igher, mayor). Cuando {\tt C} está activo y {\tt Z} inactivo ({\tt C} vale 1, {\tt Z} vale 0).
  \item{\tt LS} ({\tt l}ower or {\tt s}ame, menor o igual). Cuando {\tt C} está inactivo ó {\tt Z} activo ({\tt C} vale 0 ó {\tt Z} vale 1).
\end{itemize}

Por ejemplo, la instrucción {\tt beq destino\_salto} producirá un salto
a la instrucción indicada por la etiqueta {\tt destino\_salto} si y sólo
si el bit de estado cero está activo (Z=1), y en caso contrario
(Z=0) no interrumpirá el flujo secuencial de instrucciones.
Previo a un salto condicional, el registro de flags debe ser actualizado
mediante alguna instrucción aritmética ({\tt adds}, {\tt subs}, {\tt cmp},
\dots) o lógica ({\tt and}, {\tt orr}, {\tt tst}, \dots). En la mayoría
de los casos tenemos que añadir el sufijo {\tt s} a una instrucción normal
{\tt add}, para forzar que la nueva instrucción {\tt adds} actualice los flags.

El operando de la instrucción de salto puede ser una etiqueta {\tt b} que indica la
posición de memoria a saltar, o bien un registro en caso de {\tt bx}.

Un aspecto muy peculiar de la arquitectura ARM es que las llamadas a
subrutinas se hacen mediante un sencillo añadido a la instrucción de salto. La
instrucción {\tt bl} (también {\tt blx}) hace una llamada a una subrutina,
mediante un salto a la subrutina y escribiendo en {\tt lr} la dirección de la
siguiente instrucción.

\begin{lstlisting}
main:   mov     r1, #1
        mov     r2, #2
        bl      subrut
        mov     r4, #4  /* siguiente instrucción */
        ...

subrut: mov     r3, #3
        bx      lr
\end{lstlisting}

Si seguimos el flujo del programa primero cargamos {\tt r1} a 1, luego {\tt r2}
a 2 y lo siguiente que hay es una llamada a subrutina. En dicha llamada el
procesador carga en {\tt lr} la dirección de la siguiente instrucción {\tt mov r4, \#4}
y salta a la etiqueta {\tt subrut:}. Se ejecuta el {\tt mov r3, \#3} de la subrutina
y después {\tt bx lx} que vendría a ser la instrucción de retorno. Es decir, salimos
de la subrutina retomando el flujo del programa principal, ejecutando {\tt mov r4, \#4}.

Este sencillo esquema vale para un sólo nivel de subrutinas, es decir, dentro de
{\tt subrut} no podemos llamar a otra subrutina porque machacamos el valor del
registro {\tt lr}. La solución para extender a cualquier número de niveles es almacenar
el registro {\tt lr} en pila con las instrucciones {\tt push} y {\tt pop}.

\begin{lstlisting}
main:   mov     r1, #1
        mov     r2, #2
        bl      nivel1
        mov     r5, #5  /* siguiente instrucción */
        ...

nivel1: push    {lr}
        mov     r3, #3
        bl      nivel2
        pop     {lr}
        bx      lr

nivel2: mov     r4, #4
        bx      lr
\end{lstlisting}

Como véis en el último nivel ({\tt nivel2}) podemos ahorrarnos el tener que
almacenar y recuperar {\tt lr} en la pila.



\subsection{Estructuras de control de alto nivel}

En este punto veremos cómo se traducen a ensamblador las estructuras
de control de alto nivel que definen un bucle ({\tt for}, {\tt while},
\dots), así como las condicionales ({\tt if-else}).

Las estructuras {\tt for} y {\tt while} se pueden ejecutar un mínimo de 0
iteraciones (si la primera vez no se cumple la condición). La traducción de
las estructuras {\tt for} y {\tt while} se puede ver en los listados
\ref{lst:codigoPract2-1} y \ref{lst:codigoPract2-2}.

Para programar en ensamblador estas estructuras se utilizan instrucciones
de salto condicional. Previo a la instrucción de salto es necesario evaluar
la condición del bucle o de la sentencia {\tt if}, mediante instrucciones
aritméticas o lógicas, con el fin de actualizar los flags de estado.

\begin{lstlisting}[caption={Código del programa tipos1.c},label={lst:codigoPract2-1},escapeinside={@}{@}]
  int vi, vf, i;

  for ( i= vi; i<=vf; i++ ){ 
    /*cuerpo del bucle*/
  }

  i= vi; 
  while ( i<=vf ){ 
    /*cuerpo del bucle*/
    i++; 
  }
\end{lstlisting}

\begin{lstlisting}[caption={Traducción de las estructuras {\tt for} y {\tt while}.
Hemos supuesto que el valor inicial está en la variable {\tt vi}
y el valor final en  la variable {\tt vf} y se ha utilizado el
registro {\tt r1} como índice de las iteraciones {\tt i}.},label={lst:codigoPract2-2},escapeinside={@}{@}]
        ldr     r1, =vi
        ldr     r1, [r1]
        ldr     r2, =vf
        ldr     r2, [r2]
bucle:  cmp     r1, r2
        bhi     salir
        /* cuerpo
           del
           bucle  */
        add     r1, r1, #1
        b       bucle
salir:
\end{lstlisting}

\subsection{Compilación a ensamblador}

Para acabar la teoría veamos cómo trabaja un compilador de C real. 
\chapterend{}

%%%%%%%%%%%%%%%%%%%%%%%%%%%%%%%%%%%%%%%%%%%%%%%%%%%%%%%%%%%%%%
\end{document}