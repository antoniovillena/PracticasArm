\chapterbeginx{Presupuesto}

El presupuesto que mostramos a continuación es haciendo un pedido de 30 unidades, que
son las necesarias para cubrir los puestos del laboratorio. En la tabla ponemos
el precio unitario, para que sea fácil extrapolar los datos a otras situaciones. Cada
puesto consta de un PC, con monitor, teclado y ratón conectado en una red local y
con Linux instalado.

\begin{longtable}{ p{6cm} | p{4cm} | p{3cm}}
\hline
{\bf Componente} & {\bf Tienda} & {\bf Precio} \\ \hline
Raspberry Pi Modelo A+ & \href{http://es.rs-online.com/web/p/kits-de-desarrollo-de-procesador-y-microcontrolador/8332699/}{RS Online} & 17,26 € \\
USB-Serie con DTR & \href{http://www.ebay.com/itm//400565980256}{Ebay} & 1,44 € \\
PCB placa auxiliar & \href{http://seeedstudio.com/service/index.php?r=pcb}{Seeedstudio} & 0,20 € \\
Altavoz & \href{http://www.ebay.com/itm/261583913099}{Ebay} & 0,08 € \\
Array resistencias & \href{http://www.aliexpress.com/item//729138245.html}{Aliexpress} & 0,04 € \\
2 pulsadores & \href{http://www.ebay.com/itm/261621014025}{Ebay} & 0,02 € \\
6 LEDs & \href{http://www.ebay.com/itm/281410599537}{Ebay} & 0,17 € \\
Conector hembra & \href{http://www.ebay.com/itm/271427325429}{Ebay} & 0,06 € \\ \hline
Total &  & 19,27 € \\ \hline

\caption{Presupuesto unitario por puesto}
\label{tab:presupuesto}
\end{longtable}

  
\chapterend
