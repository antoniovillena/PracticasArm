\documentclass[12pt,a4paper]{book} % article, report, book.
\input{A1.Preambulo.paquetes.tex}
\input{A2.Preambulo.commandos.tex}
\input{A3.Preambulo.entornos.tex}
\input{A4.Preambulo.estilodepagina.tex}
\usepackage{epstopdf}
\begin{document}

\chapterbegin{Subrutinas y paso de parámetros}
\label{chp:Subrut}
\minitoc

{\bf Objetivos}: En esta sesión experimentaremos con las
subrutinas. Veremos en qué consiste la convención AAPCS y cómo
aplicarla tanto para llamar a funciones externas como para crear
nuestras propias funciones. Escribiremos un programa en C que llame
a funciones escritas en ensamblador. Por último explicaremos qué
son los registros de activación y cómo aplicarlos para almacenar
variables locales.

\section{Lectura previa}

\subsection{La pila y las instrucciones {\tt ldm} y {\tt stm}}

Se denomina pila de programa a aquella zona de memoria, organizada de
forma LIFO ({\it Last In, First Out}), que el programa emplea
principalmente para el almacenamiento temporal de datos.
%
Esta pila, definida en memoria, es fundamental para el funcionamiento
de las rutinas%
\footnote{
En este texto usaremos el término rutina (o subrutina) 
como la implementación
a bajo nivel de lo que en alto nivel se conoce como procedimientos
y funciones. La diferencia entre procedimiento y función, radica
en que las funciones proporcionan un valor de retorno. 
}%
, aspecto que se desarrollarán en esta
práctica.

El puntero de pila es {\tt r13} aunque por convención nunca se emplea esa
nomeclatura, sino que lo llamamos {\tt sp} ({\bf s}tack {\bf p}ointer o puntero
de pila). Dicho registro apunta siempre a la palabra de memoria que corresponde
a la cima de la pila (última palabra introducida en ella).

La pila tiene asociadas dos operaciones: {\tt push} (meter un elemento en la pila) y
{\tt pop} (sacar un elemento de la pila). En la operación {\tt push} primero
decrementamos en 4 (una palabra son 4 bytes) el registro {\tt sp} y luego escribimos
dicho elemento en memoria. Decimos que la pila crece hacia abajo.

\begin{lstlisting}[caption={Operación push},escapeinside={@}{@}]
        sub     sp, sp, #4
        str     r0, [sp]
\end{lstlisting}

Para sacar elementos de la pila tenemos la operación {\tt pop}, que primero extrae
el elemento de la pila y luego incrementa el puntero (decrece hacia arriba).

\begin{lstlisting}[caption={Operación pop},escapeinside={@}{@}]
        ldr     r0, [sp]
        add     sp, sp, #4
\end{lstlisting}

Para el procesador ARM estas instrucciones no existen. Sin embargo tenemos las
instrucciones {\tt stm} y {\tt ldm} que son mucho más potentes y el ensamblador
permite las pseudoinstrucciones {\tt push} y {\tt pop} que de forma transparente
traducirá a {\tt stm} y {\tt ldm}.

Un uso muy común de la pila es salvaguardar una serie de registros, que queremos
usar para hacer las operaciones que necesitemos pero que al final tenemos que
restaurar a sus valores originales. En un procesador típico escribiríamos algo así.

\begin{lstlisting}
        push    r1
        push    r2
        push    r4
        /* código que modifica los
           registros r1, r2 y r4   */
        pop     r4
        pop     r2
        pop     r1
\end{lstlisting}

Observa que el orden de recuperación de registros es inverso al de guardado. Pues
bien, en ARM lo tenemos mucho más fácil. Gracias a las instrucciones de carga/escritura
múltiple podemos meter los registros en una lista, empleando una única instrucción.

\begin{lstlisting}
        push    {r1, r2, r4}
        /* código que modifica los
           registros r1, r2 y r4   */
        pop     {r1, r2, r4}
\end{lstlisting}

En este caso el orden no es relevante, el procesador siempre usa el orden ascendente para
el {\tt push} y el descendente para el {\tt pop}, aunque nosotros por legibilidad siempre
escribiremos los registros en orden ascendente.

Las instrucciones {\tt ldm} y {\tt stm} tienen la siguiente sintaxis.

\begin{lstlisting}
    ldm{modo_direc}{cond} rn{!}, lista_reg
    stm{modo_direc}{cond} rn{!}, lista_reg
\end{lstlisting}

{\bf modo\_direc}
\begin{itemize}
  \item{\tt ia} Incrementa dirección después ({bf a}fter) de cada transferencia. Es el modo por defecto
                en caso de omitirlo.
  \item{\tt ib} Incrementa dirección antes ({bf b}efore) de cada transferencia.
  \item{\tt da} Decrementa después de cada transferencia.
  \item{\tt db} Decrementa antes de cada transferencia.
\end{itemize}

{\bf cond} Es opcional, son las mismas condiciones de los flags que vimos en el capítulo
          anterior, que permiten ejecutar o no dicha instrucción. En la práctica sólo las
          usamos para los saltos condicionales.

{\bf rn} Es el registro base, el cual apunta a la dirección inicial de memoria donde
          se hará la transferencia. El más común es {\tt sp} ({\tt r13}), pero puede
          emplearse cualquier otro.

{\bf !} Es un sufijo opcional. Si está presente, actualizamos {\tt rn} al final de la operación.

{\bf cond} Una lista de uno o más registros, que serán leídos o escritos en memoria. La
          lista va encerrada entre llaves y separada por comas. También podemos
          usar un rango de registros. En este ejemplo se almacenan los registros {\tt r3, r4,
          r5, r6, r10 y r12}.

\begin{lstlisting}
        stmdb   r1!, {r3-r6, r10, r12}
\end{lstlisting}

Si tenemos en cuenta que {\tt push} predecrementa, que {\tt pop} postincrementa y que ambas
actualizan el registro base, la traducción de las pseudoinstrucciones {\tt push \{r4, r6\}}
y {\tt pop \{r4, r6\}} serían respectivamente.

\begin{lstlisting}
    stmdb sp!, {r4, r6}    /* push */
    ldmia sp!, {r4, r6}    /* pop  */
\end{lstlisting}

Nosotros sin embargo emplearemos los nemónicos {\tt push/pop}, mucho más fáciles de
recordar.

\subsection{Convención AAPCS}


\chapterend{}

%%%%%%%%%%%%%%%%%%%%%%%%%%%%%%%%%%%%%%%%%%%%%%%%%%%%%%%%%%%%%%
\end{document}