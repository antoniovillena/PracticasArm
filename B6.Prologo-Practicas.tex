\chapterbeginx{Prólogo}

El minicomputador Raspberry Pi es una placa del tamaño de una tarjeta de crédito y un
precio de sólo 30€. El objetivo principal de sus creadores, la Fundación Raspberry Pi,
era promover la enseñanza de conceptos básicos de informática en los colegios e institutos.
Sin embargo, ha terminado convirtiéndose también en un pequeño computador de bajo
coste que se destina a muy diversos usos: servidor multimedia conectado al televisor,
estación base para domótica en el hogar, estaciones meteorológicas, servidor de discos
en red para copias de seguridad, o como un simple ordenador que puede ejecutar
aplicaciones de internet, juegos, ofimática, etc. Esto ha llegado a ser así gracias
a un vertiginoso crecimiento de la comunidad de desarrolladores para Raspberry Pi,
y que estos han explorado casi todas las posibilidades para sacar el máximo partido
de este ordenador de 30€. Esa gran funcionalidad y el bajo coste constituyen el
principal atractivo de esta plataforma para los estudiantes. Sin embargo, para los
docentes del Dept. de Arquitectura de Computadores, la Raspberry Pi ofrece una excusa
perfecta para hacer más amenos y atractivos conceptos a veces complejos, y a veces
también áridos, de asignaturas del área.

Este trabajo se enmarca dentro del Proyecto de Innovación Educativa PIE13-082, ``Motivando
al alumno de ingeniería mediante la plataforma Raspberry Pi'' cuyo principal objetivo es
aumentar el grado de motivación del alumno que cursa asignaturas impartidas por el
Departamento de Arquitectura de Computadores. La estrategia propuesta se apoya en el
hecho de que muchos alumnos de Ingeniería perciben que las asignaturas de la carrera
están alejadas de su realidad cotidiana, y que por ello, pierden cierto atractivo. Sin
embargo, bastantes de estos alumnos han comprado o piensan comprar un minicomputador
Raspberry Pi que se caracteriza por proporcionar una gran funcionalidad, gracias a
estar basado en un procesador y Sistema Operativo de referencia en los dispositivos
móviles. En este proyecto proponemos aprovechar el interés que los alumnos ya
demuestran por la plataforma Raspberry Pi, para ponerlo a trabajar en pro del
siguiente objetivo docente: facilitar el estudio de conceptos y técnicas impartidas
en varias asignaturas del Departamento. Cuatro de estas asignaturas son:

\begin{itemize}
  \item{Tecnología de Computadores:} Asignatura obligatoria del módulo de Formación
        Común de las titulaciones de Grado en Ingeniería Informática, Grado en
        Ingeniería de Computadores y Grado en Ingeniería del Software. Es una asignatura
        que se imparte en el primer curso.
  \item{Estructura de Computadores:} Asignatura obligatoria del módulo de Formación
        Común de las titulaciones de Grado en Ingeniería Informática, Grado en
        Ingeniería de Computadores y Grado en Ingeniería del Software. Es una asignatura
        que se imparte en el segundo curso.
  \item{Sistemas Operativos:} Asignatura obligatoria del módulo de Formación Común de las
        titulaciones de Grado en Ingeniería Informática, Grado en Ingeniería de
        Computadores y Grado en Ingeniería del Software. Se imparte en segundo curso.
  \item{Diseño de Sistemas Operativos:} Asignatura obligatoria del módulo de Tecnologías
        Específicas del Grado de Ingeniería de Computadores. Se imparte en tercer curso.
\end{itemize}

En esas cuatro asignaturas, uno de los conceptos más básicos es el de gestión de
interrupciones a bajo nivel. En particular, en Estructura de Computadores, esos
conceptos se ilustraban en el pasado mediante prácticas en PCs con MSDOS y programación
en ensamblador, pero el uso de ese sistema operativo ya no tiene ningún atractivo y
además crea problemas de seguridad en los laboratorios del departamento. Sin embargo,
la plataforma Raspberry Pi se convierte en una herramienta adecuada para trabajar a
nivel de sistema, es económica y ya disponemos de unidades suficientes para usarlas
en los laboratorios (30 equipos para ser exactos).
 
El principal objetivo de este trabajo es la creación de un conjunto de prácticas enfocadas
al aprendizaje de la programación en ensamblador, en concreto del ARMv6 que es el
procesador de la plataforma que se va a utilizar para el desarrollo de las prácticas,
así como al manejo a bajo nivel de las interrupciones y la entrada/salida en dicho procesador.
El aprendizaje del lenguaje ensamblador del procesador ARM usado en la
Raspberry Pi se puede completar leyendo la documentación disponible en
\cite{DATH} y haciendo los tutoriales de \cite{TASM}.  Para la parte
más centrada en el hardware también se puede consultar la amplia
documentación disponible en internet, como por ejemplo los tutoriales
disponibles en \cite{BKPI} y la descripción los modos de operación de
los periféricos conectados al procesador ARM \cite{ARMP}.


La presente memoria está dividida cinco capítulos y cuatro
apéndices. De los 5 capítulos, el primero es introductorio. Los dos
siguientes se centran en la programación de ejecutables en Linux,
tratando las estructuras de control en el capítulo 2 y las subrutinas
(funciones) en el capítulo 3. Los dos últimos capítulos muestran la
programación en Bare Metal, explicando el subsistema de entrada/salida
(puertos de entrada/salida y temporizadores) de la plataforma
Raspberry Pi y su manejo a bajo nivel en el capítulo 4 y las
interrupciones en el capítulo 5.  En los apéndices hemos añadido
aspectos laterales pero de suficiente relevancia como para ser
considerados en la memoria, como el apendice A que explica el
funcionamiento de la macro ADDEXC, el apéndice B que muestra todos los
detalles de la placa auxiliar, el apéndice C que nos enseña a agilizar
la carga de programas Bare Metal y por último tenemos el apéndice D,
que profundiza en aspectos del GPIO como las resistencias
programables.

